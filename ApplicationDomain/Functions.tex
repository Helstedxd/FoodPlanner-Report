%Funktioner - beskrivelse af systemets funktionalitet se kap 7
%	Komplet funktionsliste
%	Specifikation af funktioner
\section{Function definitions}
	The diagrams in \cref{Usage} contain different functions these functions have been given a name and listed in the following tables, where they have been assigned a complexity of Simple, Medium or Complex, and they have been assigned one or more of the following types; Read, Update, Calculate or Signaling. The more complex functions are further elaborated under each table.
\begin{table}[H]
	\centering
	\caption{Shopping list}
	\begin{tabular}{|l|l|l|}\hline
		\textbf{Function}&\textbf{Complexity}&\textbf{Type}\\\hline
	  Update shopping list  &  Complex & Update \\\hline
	  View shopping list    &  Simple  & Read   \\\hline
  \end{tabular}
  \begin{flushleft}
    \textbf{Update shopping list:} The update function will be triggered by the signal function from the inventory "Update shopping list" function. When a meal is added or removed from the food plan. The shopping list should represent what ingredients that is needed to prepare the meals in the foodplan.
  \end{flushleft}
	\caption{Food plan:}
  \begin{tabular}{|l|l|l|}\hline
		\textbf{Function}&\textbf{Complexity}&\textbf{Type}\\\hline
	  Add meal              &  Complex & Calculate, update \\\hline
	  Remove meal           &  Simple  & Update            \\\hline
	  View food plan        &  Simple  & Read              \\\hline
  \end{tabular}
  \begin{flushleft}
    \textbf{Add meal:} The add meal function is given the complex status as it involves many actions in order to add the meal to the schedule. When the user goes trough the steps necessary to schedule a meal, the system will do a lookup in order to create a list of needed ingredients. This list will be compared with what the user has in the inventory, and any ingredients that is not present on the list will then be added to the shopping list. This is were the previously described function comes into play.
  \end{flushleft}
	\caption{Inventory:}
  \begin{tabular}{|l|l|l|}\hline
		\textbf{Function}&\textbf{Complexity}&\textbf{Type}\\\hline
	  View inventory        &  Simple  & Read   \\\hline
	  Update shopping list  &  simple  & Signal \\\hline
	  Add Item              &  Complex & Update \\\hline
	  Edit Item             &  Complex & Update \\\hline
	  Remove Item           &  Simple  & Update \\\hline
  \end{tabular}
  \begin{flushleft}  	  
    \textbf{Add item:} The add item function have been marked as complex. The function is complex as there are many steps involved when adding an item, see \Cref{InvDesc} for a description of these steps. By having this function the user is able to add food to their storage manually.
    
    \textbf{Edit item:} The edit item function is complex as well. If the user changes the expiration date of an item, it could potentially update the inventory as an item can "unexpire". The same goes for changing the quantity of an food item. If the quantity reaches zero the item will be deleted, and therefore an update in the storage will be needed.
  \end{flushleft}
\end{table}
