\subsection{Activities}
To see, in which context the program will be used, we first look at the activities associated with using the program. We have chosen to split the activities in two, what happens in the home and what happens outside the home.

In the home:
\begin{itemize}
\item The program will be used to manage the food supply, by;
	\begin{itemize}
		\item Viewing a list of the stored groceries
		\item Updating the list, by adding new groceries and removing groceries, that could have been used, or groceries that has gone bad
		\item Create a shopping list of the groceries needed for the planned meals
		\item surf for recipes
		\item add new or modified recipes
	\end{itemize}
	Furthermore it should be in the home that settings are set, which could be groceries you want to ignore, because;
	\begin{itemize}
		\item You do not like it
		\item Are allergic
		\item The grocery is not associated with a certain diet
		\item E.g.
	\end{itemize}
	Certain recipes will require the use of certain kitchen tools, therefore you would want to;
	\begin{itemize}
		\item View what tools are needed
		\item Be able to check if you have the tools
		\item Associate tools with a specific recipe
	\end{itemize}
	\item While cooking the meal, you would want to follow a recipe while cooking, which mean you might have to interact with the program during the cooking.
\end{itemize}

Outside the home:
\begin{itemize}
\item While shopping for groceries, it is necessary to keep track of the shopping list, to see what should be bought.
\item If a new recipe is wanted while on the fly, it changes to the shopping list must be done.
\end{itemize}

