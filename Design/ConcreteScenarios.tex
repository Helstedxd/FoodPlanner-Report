\subsection{Concrete Scenarios}\label{ConcreteScenarios}
This section will use the process of specifying design constraints of the conceptual scenarios in \cref{conceptualScenarios} and produce a number of concrete scenarios. Each concrete scenario has a PACT overview followed by a detailed scenario description, where notes are marked by a set of parentheses, with roman numerals, e.g. (I), these notes are then listed below the scenario, and explained when in the scenario, different design choices can be made.

\subsubsection{Concrete Scenario 1: Shopping without a written list}\label{ConcreteScenario1}

\emph{PACT overview}
\begin{itemize}
\item \textbf{People:} Peter, works at a bank office.
\item \textbf{Activities:} Selecting a meal, creating a shopping list and updating inventory.  
\item \textbf{Context:} In a car and in the local supermarket.
\end{itemize}

\emph{Introduction:} Peter is on his way to the local grocery store. On the way he thinks of what he wants for dinner and from this he assembles a list of ingredients that he needs to buy. When he walks around the store picking each item he imagines crossing it off a list. On his walk around the store he picks up a few extra items that he thinks are needed at home.

\emph{Scenario}
\begin{enumerate}
\item Peter has just gotten of work and is on his way home in his car. While he drives through traffic he thinks of what meal he wants to eat. He is visited by his sister Laura and her boyfriend John, so he wants to make a good meal that requires good and fresh ingredients. He knows that Laura likes chicken(I). Peter therefore decides to buy a large chicken and ingredients for a salad. 
\item In the store Peter starts by walking over to the vegetable area and begins to check off what he needs to make a good salad(II). He is insecure of what he has at home so he buys all of the ingredients just to be sure(III). He now glances over the vegetables on the shelves and feels like taking a few apples(IV). He Looks down into his basket and checks what he has gotten so far, which is enough for a salad and a few apples for the next days.
\item Peter now goes directly over to the frozen products and picks out a chicken. Looking over the groceries he decides to also buy food for tomorrow(V). He picks up a bag of frozen potatoes and some steaks. He decides not to buy any more because he thinks there will be some leftovers from tonight's meal.  
\end{enumerate}

\emph{Design Notes}

\begin{enumerate} [(I)]
\item How does the program enable Peter to specify and find the right recipe that he wants?
\item Should the user cross off items on the shopping list or how is it indicated that an ingredient is bought?
\item  How does the program make is possible for users to see what items that they have at home so they avoid being in doubt?
\item Can the user add items via the shopping list or should they be added directly to the inventory?
\item How does the program ease the planning of a meal at any time e.g. by minimizing the number of steps to plan a meal?
\end{enumerate}

\subsubsection{Concrete Scenario 2: Forgetting to buy ingredients}\label{ConcreteScenario2}

\emph{PACT overview}
\begin{itemize}
\item \textbf{People:} Jenny, working as a teacher.
\item \textbf{Activities:} Shopping, checking shopping list and updating shopping list. 
\item \textbf{Context:} In a grocery store and in the home.
\end{itemize}

\emph{Introduction:} Jenny is out shopping for the next two days with her children. She wants to shop for lasagne and pizza. When she gets home after shopping she discovers that some ingredients are still missing. She drives back to the grocery store and buys the missing ingredients. 

\emph{Scenario}
\begin{enumerate}
\item Jenny is going to the grocery store with her two children James and Emily. She has not written a shopping list but she does know what meals to buy ingredients for. She wants to make lasagne this evening and tomorrow she and her children are going to make pizza together. 
\item In the store she walks around picking up items and checking them off of her list(I). She starts with buying the ingredients she needs for lasagne and then goes on to buy the ingredients to make pizza. Her children chose some ingredients they want to put on the pizzas(II) and they are put in the basket with everything else. After buying all the items(III) she can think of, for her planned meals, she and the children walk to the checkout. 
\item In the car on the way home Jenny talks with her children about their day. The children have been in the local sports center so they are very hungry tonight. Jenny quickly unloads the car as they arrive at home. In the kitchen she places the bought ingredients(IV). As she opens the refrigerator and puts in the new ingredients she sees that there is no cheese for the pizzas. She did not buy any when she was in the grocery store because she thought there were plenty at home(V). She tells her children to stay home and drives back to the grocery store.
\item In the store she goes directly to the dairy area and picks up the cheese she needs. Before she leaves for the check out she thinks of anything that could still be missing. She is not sure there are enough tomato sauce for the pizzas so she picks some extra cans. Now she is sure there is no more ingredients missing and walks up to the checkout. She pays for the ingredients and drives home to cook dinner with her children.  
\end{enumerate}

\emph{Design Notes}

\begin{enumerate} [(I)]
\item Should the user cross off items on the shopping list or how is it indicated that an ingredient is bought?  
\item Can the user add/edit recipes in the program with their own variants of the meal?
\item Should the user cross off items on the shopping list or how is it indicated that an ingredient is bought? 
\item Can the user add items via the shopping list or should they be added directly to the inventory?
\item How does the program make it possible for users to see what items they have at home so they avoid being in doubt?
\end{enumerate}

\subsubsection{Concrete Scenario 3: Receiving text messages to shop after} \label{ConcreteScenario3}

\emph{PACT overview}
\begin{itemize}
\item \textbf{People:} Casper, taxi driver.  
\item \textbf{Activities:} Shopping and texting with his wife.
\item \textbf{Context:} In the grocery store.
\item \textbf{Technology:} Smartphone.
\end{itemize}

\emph{Introduction:} Casper is out shopping for the next couple of days. He does not have a full list of all the ingredients that is needed at home. His wife Sandra is texting him and helping him to shop all ingredients that are needed at home. In the supermarket Casper walks around completing his list and updating with what he is told by his wife.  

\emph{Scenario}
\begin{enumerate}
\item It is weekend and Casper is in the local supermarket. He has a list of items on his cellphone that he needs to buy(I). Before he starts to shop he texts his wife and asks her if there is anything missing at home that he has to buy(II). He first collects the ingredients listed on his phone.   
\item  When he is about halfway through the list, he receives a text from his wife telling him to buy some milk, fruits, eggs, sausage and bread. He adds these items to his shopping list(III). As he walks around collecting each ingredient he checks off each ingredient(IV). He spots some meat on sale and even though it is actually not on the list he decides to buy it anyway.  
\item After collecting each ingredient on the list he walks up to the check out. Just before he gets in line his wife texts him and asks if he can get some rice. Luckily he is still in the supermarket so it is not a problem to get the rice.  
\item He gets the rice and now has everything that is needed at home. If there is still something missing, it will be bought next time he goes shopping.
\end{enumerate}

\emph{Design notes}

\begin{enumerate} [(I)]
\item What information is shown to each ingredient on the list and how is it presented?
\item How would the user know what is missing at home if the wife could not check up on it?
\item How does the user add ingredients to the shopping list or inventory?
\item Should the user cross off items on the shopping list or how is it indicated that an ingredient is bought? 
\end{enumerate}

\subsubsection{Concrete Scenario 4: Sending text messages to shop after} \label{ConcreteScenario4}

\emph{PACT overview}
\begin{itemize}
\item \textbf{People:} Olivia, housewife.  
\item \textbf{Activities:} Checking inventory and making a shopping list.
\item \textbf{Context:} At home.
\item \textbf{Technology:} Smartphone.  
\end{itemize}

\emph{Introduction:} Olivia is in her home when she gets a text from her husband asking if they are missing some ingredients. She looks at his list and checks if they have any of it at home. She also checks if there is anything not on the list that is needed. She texts her husband back with a few ingredients that he can add to his list.

\emph{Scenario}
\begin{enumerate}
\item Olivia is sitting at home and working in front of her PC when she gets a text from her husband. He asks if she can check if they are missing anything from the shopping list he has created at home. She starts by walking out to the kitchen and checks the refrigerator, the cabinets and their freezer.
\item By looking at what is in the refrigerator she sees that they could restock on milk(I). She writes it down in a text. Next she goes through the cabinets looking if they got any ketchup, bread, sugar and cereal. There is no ketchup and only half a pack of cereal. She writes these ingredients in the text as well(II). After looking in the freezer she writes that he can buy some frozen potatoes, vegetables and some ice cream. She now sends the text to her husband.  
\item After sitting down in front of her PC she remembers that they are getting guests tomorrow. They need some lunch to serve. She writes that the husband should also get some fish, fresh potatoes and fresh vegetables(III). She puts the phone away and starts working again. 
\end{enumerate}

\emph{Design notes}

\begin{enumerate} [(I)]
\item Should the user input anything else than name and quantity of the ingredient when adding it to the shopping list? 
\item Can multiple ingredients be added to the shopping list at the same time?  
\item Could the user pick a recipe and then the ingredients will be added to the shopping list?
\end{enumerate}

\subsubsection{Concrete Scenario 5: Shopping in multiple stores} \label{ConcreteScenario5}

\emph{PACT overview}
\begin{itemize}
\item \textbf{People:} Dennis, train conductor. 
\item \textbf{Activities:} Shopping and editing shopping list.
\item \textbf{Context:} In a grocery store.
\item \textbf{Technology:} Smartphone.
\end{itemize}

\emph{Introduction:} Dennis has a list of groceries he needs to buy. He knows that one store will have a sale but the other ones will not. He decides to go to the store with the sale and buy all the groceries he can get. All the groceries that he cannot buy at the first store will be bought at another one.

\emph{Scenario}
\begin{enumerate}
\item Dennis has written a shopping list for the next four days(I). The list is quite long so he does not mind shopping in different stores or on two consecutive days. He can see in the sale magazines that one of the local grocery stores has a sale today. He decides to go there and buy as many of the groceries from his list as he can(II). The rest of the groceries will be bought in another store that he usually shops in.    
\item Later the same day he drives to the store with sales. He looks at his list and starts going around checking groceries of the list(III). As he goes around he sees some sale offers that he just cannot resist and puts into his basket(IV). When he comes up to the checkout counter he looks at his list and double checks that he has crossed of and added the extra groceries. 
\item Dennis buys the groceries(V) and heads home. At home he unpacks and makes a new list of the items he is missing. The day after he goes to another local store with the intention of buying all the groceries on his shopping list. He buys all the groceries at the store without adding any impulse buys. He pays for the groceries and drives home to cook dinner.
\end{enumerate}

\emph{Design notes}

\begin{enumerate} [(I)]
\item Can the user specify how many days to shop for?
\item How does the user buy some groceries on the shopping list but not all.
\item Should the user cross off items on the shopping list or how is it indicated that an ingredient is bought?
\item Should the user input anything else than name and quantity of the ingredient when adding it to the shopping list?
\item How is it indicated that an ingredient is bought?
\end{enumerate}

\subsubsection{Concrete Scenario 6: Using a written list to shop from} \label{ConcreteScenario6}

\emph{PACT overview}
\begin{itemize}
\item \textbf{People:} Jack, works at a sawmill.
\item \textbf{Activities:} Planning meals and creating a shopping list.
\item \textbf{Context:} At home and in a grocery store.
\item \textbf{Technology:} Pen and paper.
\end{itemize}

\emph{Introduction:} Jack is sitting at home in his living room and planning three days ahead. He is first planning the meals and then he is assembling all the necessary ingredients on a shopping list. 

\emph{Scenario}
\begin{enumerate}
\item Jack is sitting in his sofa with the TV turned on. As usual Jack likes to sit down and plan his meals ahead of time. He wants to plan the following three days, and starts by planning the meal for tomorrow(I).
\item He wants to cook a good meal because his sister is visiting. He knows that she loves lasagne(II) so that is what he will cook. He writes lasagne on his list and as bullet points he writes all the ingredients he needs(III). He does not write from a recipe but only from what he remembers has to be put in a lasagne.
\item The following day he is not sure what he wants. He wants to make something with fish but he does not know if he wants to have rice or noodles with it(IV). He walks into his kitchen and checks what he has. He cannot find any rice or noodles(V), but decides to make fish with noodles. He adds fish, noodles, vegetables and some spices to his list.
\item On the last day he wants to cook a big meal so he can use the leftovers the following days. He decides to make pork chops with rice. He knows that there is no rice at home so he writes that on the list. He now needs to add the pork chops, cream, milk, paprika and asparagus to the list because he knows that he does not have those at home. 
\item The meal plan now consists of lasagne, fish and pork(VI). The shopping list has all the items needed to make these meals. The next couple of hours Jack will add items to the list when he discovers what else he needs.   
\end{enumerate}

\emph{Design notes}
\begin{enumerate} [(I)]
\item How does the user add a meal to a specific day in his meal plan?
\item How does the user search for specific recipes?
\item Are the ingredients for the meal added to the shopping list automatically?
\item Can the user search for different parameters in a search?
\item Can the user see if he is missing ingredients for a recipe?
\item Where can the user see an overview of his plan?
\end{enumerate}

\section{Concrete scenarios evaluation}
This section will present the results that has been deducted from evaluating the different scenario notes and comparing these to one another. The intentions with performing an evaluation of the scenario notes is to uncover design considerations that has to be addressed. The section is structured with each aspect of the program divided into subsections. Each subsection has some associated windows and each window has design considerations. Each window has some shared design considerations such as layout of information, navigation through information and relevance of information. This is common for all the screens and will be described as little as possible.

\subsection{Start}
This aspect is implemented in the \textbf{Start screen} and is the first screen a user will encounter. It has to show the user different icons that guides the user into different aspects of the program. The different functionalities for the start screen are as follows.

\begin{itemize}
	\item Start screen.	
		\subitem Enable navigation to Meal calender screen, Recipe browsing screen, Settings screen, 				Shopping list screen, Inventory list screen and Login screen.
\end{itemize}

\subsection{Meal schedule}
This aspect includes the \textbf{Meal schedule screen} and \textbf{Specified day screen}. Meal schedule is a calender like overview of the user's meal plan. It includes planned and unplanned days. The specified day screen is only for one day and shows all the scheduled meals for the specified day.

\begin{itemize}
	\item Meal schedule screen.
		\subitem What information needs to be displayed? 
		\subitem Indication of news/ changes.
	\item Specified day screen.
		\subitem Add meal button.
		\subitem View meals.
		\subitem Change meals.
\end{itemize}

\subsection{Recipe}
This aspect includes the \textbf{Browsing recipes screen} and \textbf{Specific recipe screen}. The first screen is used by the user to browse different categories which gives the user an easy overview of these. The other screen is used to display specific recipes and information such as ingredients, description and cooking guide.

\begin{itemize}
	\item Browsing recipes screen.
		\subitem Different categories. 
		\subitem Indication of recipes that the user has ingredients for.
		\subitem Way of browsing.
		\subitem Search function.
	\item Specific recipe screen.
		\subitem Relevant information about the recipe.
		\subitem Add to meal plan method e.g. a button.
\end{itemize}

\subsection{Login}
This is a single screen that is used the first the a user starts the program and whenever a user wants to relog.

\begin{itemize}
	\item Loading screen.
		\subitem What information should this screen show/ ask for?
\end{itemize}

\subsection{Settings}
This aspect of the program deals with displaying and enabling the user to modify settings in the program. This aspect has the \textbf{Settings screen}, \textbf{Specified setting category} and \textbf{Change a specific setting}. It is debatable if all of these screens are necessary. This issue will be evaluated through the design process.

\begin{itemize}
	\item Settings screen. 
		\subitem What settings is there?
		\subitem Can the settings be categorized?
	\item Specified setting category.
		\subitem What information needs to be shown for each setting?
		\subitem How can the user change these settings?
	\item Change specific setting.
		\subitem How does the user save the changes?
\end{itemize}  

\subsection{Shopping list}
This aspect of the program deals with the display and administration of the shopping list via the \textbf{Shopping list screen} and \textbf{Shopping list new item screen}. These screens gives the user an overview of the shopping list and a way to add new items to the list.

\begin{itemize}
	\item Shopping list screen.
		\subitem An overview of all ingredients with relevant information.
		\subitem A way to add new items to the list.
		\subitem Ways for the user to modify specific information e.g. item quantities.
	\item Shopping list new item screen.
		\subitem Does the user need to fill in all information about a new item?
		\subitem How does the user complete the add item?
		\subitem How does the user undo information? 
\end{itemize}

\subsection{Inventory}
This aspect of the program is used by the user to administrate the inventory. This has two screens
\textbf{Inventory list screen} and \textbf{Inventory list new item screen} and is similar to the shopping list aspect. 
    
\begin{itemize}
	\item Inventory list screen.
		\subitem Overview of each item in the inventory with information about quantity, expiration date and possibly more.
		\subitem Way to add new item to the inventory
		\subitem Search function.
	\item Inventory list new item screen.
		\subitem The same considerations as the \textbf{shopping list new item screen}.
\end{itemize}


