\Section{Scenario Corpus}

In this section the scenario corpus is going to be described. The scenario corpus will be based on the user stories and the conceptional scenarios. To make the scenario corpus, the similarities of the user stories and conceptional scenarios will be looked upon and compared with the domain.

\subsection{Planning}

When looking at \ref{ConceptionalScenarioA} it can be seen that the people who are like this, do not like to organize their food for a long time, but like to plan day-for-day. This is not the case when looking at \ref{ConceptionalScenarioB} it can be seen that these people like to plan far ahead and organize for a couple of days.

Planning is also ideal to cook the food the best possible way. In section \ref{UserStories} person B stated that meat is taken from the freezer in the morning, therefore if the meat being taken from the freezer takes longer time to defrost than anticipated, the food might not be ready to cook and that will provide a problem. So to give the user the ability to look at the ingredients of the foodplan for the next day, it will be easy for the user to take the ingredients that need to defrost out of the freezer in enough time. 

\subsection{Grocery shopping}

Based on the user stories in section \ref{UserStories}, grocery shopping is most often done when people already are out of the house. So having a food inventory administration system and keeping a shopping list, makes people able to shop whenever they find it convenient. Some people might go home before shopping to check what they already have, but this is not needed if it is easy to see what you have.

\subsection{Cooking}

If cooking is done at a specific time, planning meals is also most optimal like the persons that behave like the conceptional scenario B \ref{ConceptionalScenarioB}, because you can go shopping a couple of days before you need the food, and do not have to shop every day, so if the person gets off work late one day, the shopping could have been done some days before, so the cooking will still be done, though you do not have to go shopping.

\subsection{Food inventory administration}

In \ref{ConceptionalScenarioA}, people does have a feeling of what is being needed in the home, where as the people of \ref{ConceptionalScenarioB} plan their meals from what is in the home. This is a similarity since if you have a feeling of what you have to buy, you have a feeling of what you already have. When the system keeps track of this there is no need to have a feeling, and maybe buy something you already have. If the system has a design that is easy to browse through, and find what you are looking for, the amount of items that will be bought when you already have it at home will decrease.