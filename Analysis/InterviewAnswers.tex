\subsection{Interview answers}
\subsubsection{25 year old single male\fxnote{What should we call these?}}
\begin{itemize}
  \item Sex: Male
  \item Age: 25
  \item Education: Vocational\fxnote{use this word for 'erhvervsuddannelse'?}
  \item Work: Skilled Sailor
  \item Household: 3 singles
\end{itemize}
\emph{What do you characterize as food waste, and is it a problem in the household?}

If any kind of food is thrown out, it is food waste according to the participant, furthermore he thinks food waste is a problem in the household, he elaborates that he thinks it is because some food is sold in bundles, that are too big compared with what is needed, and that they therefore buy too much food.

\emph{Is there a difference in the type of food, that is thrown out?}

It is mostly vegetables, which is thrown out in the household.

\emph{What factor is in focus, when food is thrown out?}

The participant only look at the freshness of the food, and does not take best before date in account.

\emph{Planning meals and shopping.}

The participant does rarely plan more than one day ahead, and does not plan at any specific time during day. Furthermore the participant mostly shop for what is wanted the specific day, and it is rarely considered, if a meal could be made from what is in the home. No one specific in the household do the shopping or the cooking, but they try to schedule the shopping with who has the time, or already has to be out.

\emph{What is the most important about your diet?}

Quality is the most important, closely followed by organic food, they try to avoid food waste, but it is not in focus.

\emph{Do you use leftovers?}

The participant always use leftovers, both because he finds it stupid, to throw it out, but also to save some money.

\emph{When do you shop, for how long, how much and how many times?}

Preferably in the morning, because there are less people. The shopping only takes 5-10 minutes and is done 4 - 5 times a week. Breakfast and lunch is often bought for a few days, where the dinner is mostly bought at the day it is needed.

\emph{What affects you in the shopping situation?}

The participant is not affected by sales in general, but do impulsive shopping if groceries are on sale, because they are close to the best before date.
\subsubsection{53 year old married female}
\begin{itemize}
  \item Sex: Female
  \item Age: 53
  \item Education: University\fxnote{use this word for 'erhvervsuddannelse'?}
  \item Work: Pedagogue
  \item Household: 2, one married couple
\end{itemize}
\emph{What do you characterize as food waste, and is it a problem in the household?}

Sales that requires you to buy to large quantities, which makes you buy to much, is seen as food waste, together with bad usage of leftovers. Food waste is not seen as a problem in this household.

\emph{Is there a difference in the type of food, that is thrown out?}

Even though food waste is not seen as a problem, it rarely happens that bread is thrown out, this is because the packages the food is sold in, are to big.

\emph{What factor is in focus, when food is thrown out?}

The participant only look at the freshness of the food, and does not take best before date in account.

\emph{Planning meals and shopping.}

The participant plan in the way that, food is taken out of the freezer in the morning, furthermore are there often made extra food, to have easy and fast prepared food, for the next few days, or to have something for a lunch, to bring to work. What needs to be shopped is mostly planned in the morning, when the food is taken out of the freezer, and the participant therefore knows what needs to be bought, if something unexpected happens, resulting in an unexpected dinner, makes shopping after work, a necessity. It is mostly the participant who does the shopping, but the other person in the household, may be asked to do some shopping.

\emph{What is the most important about your diet?}

The participant want to avoid food waste as a primary focus, next is the prize and the excitement of the meal, taken into consideration.

\emph{Do you use leftovers?}

The participant always use leftovers. And plans a dinner, so there will be leftovers, to ease the cooking for days to come.

\emph{When do you shop, for how long, how much and how many times?}

The participant prefers to shop after work. The shopping only takes about 10 minutes and is done 3-4 times a week. furthermore the participant prefers to shop for a few days only, to keep the food fresh. this results in a fridge that in quite empty, and as a result, the shopping is often a necessity.

\emph{What affects you in the shopping situation?}

The participant is affected by "decorated" sales mend for tempting to do impulsive shopping, furthermore the participant does not like to buy the food that is on sale, because it is close to the best before date.
\subsubsection{75 year old married female}
\begin{itemize}
  \item Sex: Female
  \item Age: 75
  \item Education: Vocational\fxnote{use this word for 'erhvervsuddannelse'?}
  \item Work: Pensioner
  \item Household: 2, one married couple
\end{itemize}
\emph{What do you characterize as food waste, and is it a problem in the household?}

No use of leftovers, and food that is badly stored, which makes it go bad. Food waste is not seen as a problem in the household, if there are small leftovers, the dog will get it.
%Sales that requires you to buy to large quantities, which makes you buy to much, is seen as food waste, together with bad usage of leftovers. Food waste is not seen as a problem in this household.

\emph{Is there a difference in the type of food, that is thrown out?}

The participant can not say what difference there could be, because food is never thrown out.

\emph{What factor is in focus, when food is thrown out?}

If food would be thrown out, the freshness will be the only factor.

\emph{Planning meals and shopping.}

The participant likes to shop large quantities of meat, because it is bought from a butcher who's shop is far away from home. Where items like bread and milk is bought as needed, the participant elaborates that there is rarely missing anything, because alternative ingredients will be found.

\emph{What is the most important about your diet?}

It is important that the diet if healthy, and filled with energy, the meat has to be proper, and therefor it is bought from a butcher

\emph{Do you use leftovers?}

The participant always use leftovers.

\emph{When do you shop, for how long, how much and how many times?}

There is no specific time in the day which the participant likes to shop, and the shopping takes place 2 to 3 times a week.

\emph{What affects you in the shopping situation?}

The participant likes to buy savings, and fill the freezer, and can be tempted to do impulsive shopping, furthermore the participant likes to buy the food that is on sale, because it is close to the best before date.

\subsubsection{55 year old single woman}
\begin{itemize}
  \item Sex: Female
  \item Age: 55
  \item Education: Skilled Worker\fxnote{use this word for 'faglært'?}
  \item Work: Teaching assistant
  \item Household: 2, mother and daughter
  \item Phone: Yes, android
  \item Tablet: Yes, android
\end{itemize}

\emph{What do you characterize as food waste?}

When food is thrown out in the trash. This applies to both prepared and unprepared food. If animals are fed with the food instead of it getting thrown out, it is not food waste.

\emph{Is foodwaste a problem for the household?}

No, everything edible that gets too old, it is fed to the animals of the household, more specifficaly hens, as a replacement of fodder.

\emph{Is there a difference in what gets thrown out?}

No, there is nothing specific that gets thrown out the most.

\emph{What factors are considered when throwing food out?}

Mostly the freshness of the food. Not so much the best before date, but mostly how the food looks, feels, and smells.

\emph{When do you plan your meals and shopping?}

The planning is done a 2-3 days ahead of when the food is getting eaten.

\emph{How is the shopping planned?}

Mostly out of what is wanted to eat and discounts.

\emph{How is the coordination of the members of the household?}

The mother makes everything, the daughter neither prepares the food or does any shopping.

\emph{What is valued the most about your diet?}

What the household want to eat, variation of the meals, and a little economy.

\emph{Do you use leftovers?}

Yes, leftover gets used for lunch or dinner, depending on the qunatity of the leftovers.

\emph{When do you grocery shop?}

In the morning before work, or after dinner.

\emph{How long does the grocery shopping take?}

About an hour because there is being shopped in more than one store.

\emph{How many times a week is grocery shopping done?}

3 times a week.

\emph{WHat is the quantity when shopping?}

Depends from time to time.

\emph{What affects you in the groecry shopping situation?}

The discounts are planned from home, as the participant looks through discount magzines. But if there is something practical and cheap, the particapant will buy on impulse.

\emph{How do you make your shopping list?}

The shopping list is made in by hand on a notepad, but if there was a simple sulution for tablet or smartphone, the participant would be willing to try it out. The participant have already tried using Evernote for a shopping list that would sync between different devices.