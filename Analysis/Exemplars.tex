\section{Exemplars}
In this section, the current IT-systems will be described and what other systems that can be used to plan what food to eat. we will find ideas from the systems

\subsection{Food Planner}
one technology that are currently being used to make a food plan is a mobile application called 'Food Planner'.
In this application the users can plan meals ahead of time, lookup recipes, look at what groceries that needs to be bought,
list what the user have in the fridge so the grocery list appends to the items in the fridge and more. In the following we 

\begin{figure}[H]
    \centering
    \includegraphics[width=0.5\textwidth]{Grafik/FoodPlanner/index}
    \caption{An image displaying the index of the application Food Planner}
    \label{FoodPlannerIndex}
\end{figure}

%Kan filføle flere billeder af FoodPlanner appen og uddube dem.

\subsection{Website}
Another alternative could be a tool found on the website madplanuge.dk\cite{madSpild_madPlanUge}. This tool is a basic foodplanner, where the foodplan can be planned one week ahead, but the users storage is taken into consideration.

When the user enters the website the user can either choose between 20 random recipes or search for something the user would like to eat.
Then the user can choose up to 3 recipes for each of the days so the user can plan breakfast, lunch and dinner for each specific day.
After the food plan has been selected, the user can then get a shopping list of all the needed items.
If the user creates a user on the site, the user will be able to save, and favorite different recipes.
If the user finds a recipe with some ingredients who the user likes, would the user be able to get more recipes based upon those ingredients.

\begin{figure}[H]
    \centering
    \includegraphics[width=0.5\textwidth]{Grafik/madplanuge}
    \caption{An image displaying the the website Madplan Uge}
    \label{MadPlanUge}
\end{figure}
