\subsection{Activities}
To see in which context the program will be used, we first look at the activities associated with using the program. The activities have been split into two categories, what happens \textit{inside} and \textit{outside} of the home, furthermore the activities are based on the the user stories  \cref{UserStories}, and from the information gathered in \cref{InformationGathering}, as well as requirements fount in the people part, \cref{PeoplePACT}, of this analysis.

In the home:
\begin{itemize}
\item The program will be used to manage the food supply by:
	\begin{itemize}
		\item Viewing a list of the stored groceries
		\item Updating the list, by adding or removing groceries that could have been used, or groceries that have gone bad
		\item Clearing the stored groceries. This should be done when the program is initiated for the first time or after a long period of inactivity
		\item Create a shopping list of the groceries needed for the planned meals
		\item Search for recipes
		\item Add new or modified recipes
	\end{itemize}
	\item Furthermore it should be in the home that preferences are set, for example groceries you want to ignore, because:
	\begin{itemize}
		\item You do not like them
		\item You are allergic
		\item The grocery is not associated with a certain diet
	\end{itemize}
	\item While cooking the meal, you would want to follow a recipe while cooking, which mean you might have to interact with the program during the cooking. Optimally a solution to this, would be to lessen the need of touching the devise, by having as much of the cooking information on the screen as possible, if interaction is needed, it should be done easily with as few clicks as possible, furthermore larger input areas will allow the user to f.x. use an elbow to navigate with.
\end{itemize}

Outside the home:
\begin{itemize}
\item While shopping for groceries, it is necessary to keep track of the shopping list, to see what should be bought.
\item If a new recipe is wanted on the fly, the shopping list must be updated
\item In certain circumstances it must be possible to define for how many days the shopping should be done.
\end{itemize}

Frequent activities such as looking for a recipe should be easy to do, but some activities such as clearing the list of stored groceries should be easy to learn. This could be done by having a walkthrough when a user tries to do this particular activity. The program should also have easy navigation options to allow for mistakes to happen, such as going into undesired program pages, and still be able to get back on track. If the user is interrupted and have to pause their usage of the program, the user should be able to continue later from the same point.

Searching for specific recipes can have a long response time, which could become a point of frustration for the user, especially if the search takes more than five seconds. As the recipes are available online there is a risk of high latency between a user's device and the server containing the recipes. A workaround for this could be to have a local version of the recipe database, and only have the device synchronize the two databases once in a while.