\subsection{Design Language}
This section will use the physical design process, in which the prototypes and the conceptual model are combined, to produce the design language for the program. The physical design process is concerned with defining the look and feel of the program. The process also defines the allocation of functions and the knowledge between a user and a device. The product, which is the design language, is a standardization of the interaction with the program and the physical style of the program e.g. buttons, shapes, colours.

\subsubsection{Program Screens and Functionality}
The domains was identified in \cref{ScenarioCorpus} and they are as follows: planning, shopping, cooking, inventory and general. In \nameref{Sketches} the domains were used to structure the screens in the program so that each domain had its own screen. In the \nameref{ConsMod} \cref{ConceptualModelPicture} the domains were used as main elements. Because of the structural contribution that the domains give they will be used as a basis for the actual screens of the program. The following text will describe each screen of the program, the funcionalities of the particular screen and how the functionality is available for the user.

\paragraph{Planning}


\paragraph{Shopping}

\paragraph{Cooking}

\paragraph{Inventory}

\paragraph{General}   
 

   