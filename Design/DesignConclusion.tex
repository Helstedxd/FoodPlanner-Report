\chapter{Design Summary}
In \cref{ScenarioBasedDesign}, \nameref{ScenarioBasedDesign}, the interaction design of the program was defined. This was done using the Scenario-based Model from the \textit{Design and Evaluation of User Interfaces} course. This model used real world experiences of users as \nameref{UserStories} written in \cref{UserStories}. From these, \nameref{conceptualScenarios} in \cref{conceptualScenarios} and \nameref{ConcreteScenarios} in \cref{ConcreteScenarios} were created. \nameref{ConcreteScenarios} were used to develop \nameref{Sketches} shown in \cref{Sketches}. The conceptual scenarios and Stories were processed to produce \nameref{DesignRequirements}, described in  \cref{DesignRequirements}, and \nameref{ScenarioCorpus} in \cref{ScenarioCorpus}, which were then used to define a \nameref{ConsMod} shown in \cref{ConsMod}. The paper prototypes and the \nameref{ConsMod} were then combined to formulate the \nameref{DesignLanguage}, described in \cref{DesignLanguage}, for the program. The \nameref{DesignLanguage} defined a standard for the interaction design of the program.  

In \cref{Architecture}, called \nameref{Architecture}, the \nameref{ComponentArchitecture} in  \cref{ComponentArchitecture} defined the architecture as MVVM. The \nameref{ProcessArchitecture} in \cref{ProcessArchitecture} defined the architecture, by finding which distribution pattern suited the system best, then defining how the pattern is adapted into the system. The distribution pattern became a variant of the centralized pattern. Both the component architecture and the process architecture was supported with basic knowledge by \cref{DataBaseModel/Classes}, \cref{MVVMSection} and \cref{EntityFrameworkSection}.   
