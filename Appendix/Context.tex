\subsubsection{Context}
In this section we look at the context in which certain activities are performed. The examined contexts are the physical environment, and the social context.

\paragraph{Physical environment}
Using the program in different locations will have different impact on the user experience. Having the program on a device will set some limitations, as the user will have to interact with the program, during different situations.
\begin{itemize}
\item With the households inventory stored in the program, a user will not have to be home, to see what is missing, if he/she chooses to add another recipe.
\item While shopping, one hand must be free, to check which items that has been put in the basket, and what else must be bought.
\item While cooking at home, an ingredients list could be revisited, or a cooking guide will have to be followed, with greasy hands this kind of interaction will be difficult.
\end{itemize}

\paragraph{Social context}
The program can be used in different social context.
\begin{itemize}
\item If more than one person share the same food plan, they must both have access to the program, and if it is handled on their personal devices, synchronization is necessary.
\item The program can be used by only one person and on one device only, therefore synchronization might be a nuisance instead of a trait for this type of user.
\item With an online database of recipes, it would be possible to add new recipes and share these, with other users.
\end{itemize}\fxnote{Conclude on this plz!}