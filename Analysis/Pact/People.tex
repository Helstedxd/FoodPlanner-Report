\section{PACT Analysis}
PACT is an analysis model that deals with People, Activities, Context \& Technology.
\fxnote{Give a more detailed description}

\subsection{People}
All people need food, and many also have to buy and make it themselves. There are many ways to prepare a meal, and people have different preferences of what they eat and why. In this analysis we will focus on the social differences between people and try to describe their different motives and preferences, to get an idea of who could benefit from a system, that would help them organize grocery shopping and preparing meals more efficiently.

\subsubsection{People living after a special diet, or having specific wishes for food plans}
People that are on a specific diet needs to buy food based on the diet, and prepare it in a specific way.
Examples could be:
\begin{itemize}
\item Sports performing (Bodybuilders)
\item Vegans/vegetarians
\item Organically minded
\end{itemize}

\subsubsection{People who want to save time when making and planning cooking and doing the groceries.} 
If people don't plan what they are going to eat through the week, they often have to buy groceries everyday, or maybe they start cooking something that takes long time compared to when they begin. In this case a foodplan organized based on how long time it takes to prepare a meal and what groceries you have, could help save time in the everyday life.
\begin{itemize}
\item Students
\item Parents
\item Families
\end{itemize}

\subsubsection{People who wants to be social while eating}
When people wants to get together and eat for different occasions, it might be beneficial if they could plan the meal based on the preferences of the people involved.
\begin{itemize}
\item “Social eaters”
\item Students
\item Parents
\item Living budget minded People
\item Students
\item Parents
\end{itemize}

\subsubsection{People who wants to eat together with other people of same interests and food habits}
\fxnote{Merge this with social eating section above?}
Some people might want to eat with others sharing their interests and preferences regarding food.
\begin{itemize}
\item Vegans/vegetarians
\item Organically minded
\item “Social eaters”
\end{itemize}

\subsubsection{Comparison}
It is common for all the described people that they could save time and/or money by planning their grocery-shopping and meals. A system would have to be fast and and easy to use, so that the time spent planning doesn't exceed the time it would normally take. The system should automate tasks for the user, for example by recommending recipes and keeping track of what food they already have at home. It must be very easy for the user to input the groceries, and not become a burden, preferably this should be fully automated, for example by adding groceries from a shopping list that a user used when shopping, or semi-automated by allowing the use to scan the product barcode.

% familes, compare them 

