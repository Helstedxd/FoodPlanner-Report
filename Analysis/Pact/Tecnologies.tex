\section{Technologies}

When doing a PACT analysis it is important to look at the technologies included in the design. This section is split up into 3 different parts; the input, the output, and the communication.

\subsection{Input}

Since the product is going to be released on tablets and cell phones, all the input tecnologies are incorporated in these. No external plug-in devices are going to be used for the application. The list of input technologies are:

\begin{itemize}
    \item Camera
    \item Multi-touch screen
    \item Physical buttons
\end{itemize}

The camera is used as an input technology in the sense, that it is needed to scan the barcodes of the products. When a barcode is scanned, the cell phone or tablet are supposed to recognize the product, and in that way know what product to modify in quantity, expiration date or something else.

The multi-touch screen is used to navigate in the app. If a recipe are being viewed and the users needs to zoom in on text, using two fingers to pull the recipe bigger so it is easier readable.

The physical buttons are also used to navigate throughout the program. Though the buttuns will mostly only be to view the setting, or exit the program quickly, the buttuns for modifiying the sound level of the app, is also used, when viewing a video in the application.

\subsection{Output}

The list of output technologies are:

\begin{itemize}
    \item Screen
    \item Speakers
    \item Vibration
\end{itemize}

The creen is used for displaying the app itself. By being able to see the app, the user has a easier way of navigating than if the applications just relied purely on sound. It is also why the input technology multi-touch screen are smart, because being able to see what you navigate through makes navigation in an application easier.

Speakers are used for giving letting the user achieve an audio based experience when using the application. The speakers will be used when showing video to the user, so the user can hear what is happening in the video. It might also be used for feedback when a user does something specific in the app. If the user is downloading a first recipe, the app might display a sound of achievement to let the user know that something new has been done.

The vibration is used for user feedback, when a user does something he can not like try to access a feature he can not access, a vibration will help the user understand that it can't be done at the moment. Though vibration alone does not mean much, but combines with audio or visual elements will let the user know that the feature can not be accessed.

\subsection{Communication}

Communication means hwo the application are supposed to communicate with the interet and other users. The list of technologies is:

\begin{itemize}
    \item Wi-fi
    \item Mobile Networks
\end{itemize}

The Wi-Fi of the tablet or cell phone are used for connecting to the internet, to be able to synchronize the changes in the program so it can be viewed on a different device. But the Wi-fi is also used for accessing databases so information about products can be accessed.

Mobile networks is used for the same reason as Wi-fi. Though some mobile devices does not have Mobile Networks. Some tablets are cheaper when they only provide Wi-Fi tecnology, therefore both Wi-fi and mobile networks are technologies worth considering.

\subsection{Conclusion}

Many different technologies are required to be considered when doing a PACT analysis, the technologies needed for the food planner application is from the 3 catagories; input, output, and communication. The technologies ranges from physical buttons, to the wi-fi element needed for the syncronization of the data.
