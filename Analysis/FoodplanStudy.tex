\section{Foodplan study}
\textit{Stop Spild Af Mad}\cite{madSpild_RapportAdfaerd} has conducted a study where eight informants had to create and follow a weekly food-schedule.
Each participant had to create a foodplan, in which they could freely choose the different recipes with inspiration from cookbooks and/or the internet.

\subsection*{Planning}
The participants had some trouble with pulling themselves together and getting started creating the foodplan, but they were happy to have it, once it was done. It could be beneficial for the person that were in charge of grocery-shopping to be the one making the schedule, since the involvement of other members from the family could increase the time of making it.

\subsection*{Modules}
It was almost impossible for the informants to follow the schedule for a whole week without having to reorganize or modify it. Most of the informants made small modules that could be moved around, depending on the family's situation. It was optimal with modules spanning two to three days. Varied meals was preferred, and not simply reheated meals.

\subsection*{Grocery shopping}
Seven out of the eight participants were shopping groceries almost every day. This was not seen as a nuisance, but after having planned grocery-shopping for a whole week, they experienced a big relief in their everyday life. This was expressed in terms of more personal freedom as a perk of saving time and money. Planning ahead also decreased buying on impulse.

\subsection{Reflection} \label{FoodPlanStudyReflection}
The participant had trouble getting started creating a new food plan for the week. They also wanted a more realistic food-schedule instead of having to create it from cookbooks. Realistic in the sense, that the plan should fit into their context. A solution to this problem might be solved by automating a lot of the tasks involved with having to come up with the different recipes and do the grocery shopping.
By having an application come up with suggested meals based on different aspects such as the:
\begin{itemize}
\item Food you already have
\item Food you like
\item Price
\item Stock of stores nearby
\end{itemize}
People could quickly create the schedule or maybe even have the application create it for them.