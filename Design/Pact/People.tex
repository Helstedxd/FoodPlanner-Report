\section{People}
Everyone needs food, and many also have to buy and make it themselves. There are many ways to prepare a meal, and people have different preferences of what they eat and why. In this analysis the focus will be on the social, physical and psychological differences between people and on a description of their different motives and preferences. This description will be used to get an idea of who could benefit from a system, that would help them organize grocery shopping and preparing meals more efficiently.

\subsection{Physical differences}

People have different physical abilities. Some people have bad vision, some have bad hearing, and so on with a lot of different physical conditions, and taking this into consideration is an important aspect of a PACT analysis.

Looking at the ergonomically aspects of the application, is not necessary, since the ergonomics are decided by the company who creates the device that the application run on. Therefore ergonomics are not something that can be affected by the application.

\subsection{Psychological differences}

People vary in the way that they function psychologically. But some of the common traits to look at when designing software are:

\begin{itemize}
    \item The meaning of buttons
    \item Not too hard to remember instructions
\end{itemize}

The meaning of buttons is very important because depending on where the software will be released, buttons mean different things. Since the food planner application is going to be released in Denmark, it is therefore important that the Danish people will perceive the buttons correctly.

It is also important that no instructions or commands are too long, because it will be hard to remember for some people, and when making a product you must take the weakest user into consideration when looking who to make the product for.

\subsection{Social differences}

People have different requirements of a product, and it is therefore important for the developer to make sure everyone can get, from the product, what they want. Therefore some groups of people with social differences will now be mentioned.

\subsubsection{People living after a special diet, or having specific wishes for food plans}
People that are on a specific diet needs to buy food based on the diet, and sometimes prepare it in a specific way.
Some examples of these people could be:
\begin{itemize}
\item Athletes (Bodybuilders)
\item Vegans/vegetarians
\item Organically minded
\end{itemize}

\subsubsection{People who want to save time when making and planning cooking and doing the groceries} 
If people do not plan on what they are going to eat throughout the week, they often have to buy groceries everyday, and maybe the food they are preparing takes a longer time to cook than they expected. In this case an organized foodplan based on how long time it takes to prepare a meal and what groceries you have, could help save time in the everyday life. Examples of people who can benefit from saving time because of tight schedules can be:
\begin{itemize}
\item Students
\item Parents
\item Families
\end{itemize}

\subsubsection{People who wants to be social while eating}
When people wants to get together and eat for different occasions, it could be beneficial if they plan the meal based on the preferences of the people involved. The reason why people would like to be social while eating could be just wanting to talk to others, but also to save money by preparing bigger meals, and people trying to have less food waste by cooking together. Examples of people who would like to be social while eating can be:
\begin{itemize}
\item Social eaters, people who only get together to eat a meal together
\item Students
\item Parents
\item People with a tight or small budget
\item Students
\item Parents
\end{itemize}

Some people might only want to eat with people of the same interests and food habits, for the same reasons as the people wanting to be social, but finds it easier if cooking together with people of the same interests and habits. These people include:

\begin{itemize}
\item Vegans/vegetarians
\item Organically minded
\item Social eaters
\end{itemize}

\subsubsection{Comparison}
It is common for all the described people that they could save time and/or money by planning their grocery-shopping and meals. A system would have to be fast and easy to use, so that the time spent planning does not exceed the time it would take to plan a meal completely manual. The system could automate tasks for the user, for example by recommending recipes and keeping track of what food they already have at home. It must be easy for the user to input the groceries, and not become a burden, preferably this should be fully automated, for example by adding groceries from a shopping list that a user used when shopping, or semi-automated by allowing the use to scan the product barcode.