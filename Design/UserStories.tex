\section{User Stories} \label{UserStories}
The first element of the model is to create user stories, with different situations of planning. From these conceptional scenarios can be formulated, which in turn can be used to define concrete scenarios and use cases. Furthermore these user stories are looked upon together with the conceptional scenarios, to create a understanding, which will result in requirement and problem definitions, as well as defining a scenario corpus. This first section introduces the user stories and the following sections will then be continued from here.

\subsection{Person A}
The person does not plan ahead for long periods, meaning that only 1 day is planned at a time. The planning does not take place at any specific time during the day. The shopping is typically done for the current day and it is not considered what the person has at home before buying groceries. Person A is living together with others, the task of shopping is mostly given to whoever has the time or is out already.

\subsection{Person B}
The person plans a few days at a time by cooking larger quantities, so leftovers will be available for later. On the day that the cooking is done, meat is taken out of the freezer in the morning to be prepared in the evening and if possible eaten the next days. The shopping is planned when the meat is taken from the freezer, either to supplement with missing groceries or if there is no meat, the person will have to shop, which is done after work.  
 
\subsection{Person C}
The person plans 2-3 days ahead. When the person shops he/she will buy what the person and family wants to eat, or if there are discounts at the supermarket. Discount magazines are also looked through when they arrive in the mail, so the person knows what to buy where, when going grocery shopping.

\subsection{Person D}
The person buys large quantities of meat and based on the meat the rest of the ingredients are bought, so that the recipe can be made.
