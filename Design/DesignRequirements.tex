\subsection{Design Requirements} \label{DesignRequirements}
When analyzing the interviews in \cref{InterviewAnalysis}, conceptual scenarios in \cref{conceptualScenarios} as well as the PACT analysis in \cref{PACTAnalysis}, lastly also personas, which can be found in \cref{PersonasAppendix}, some design requirements will be discovered. In this section these requirements will be presented and described, the, and later used with parts of the report, to define a scenario corpus. Furthermore the long list of requirements have been sorted into the same domains as can be found in scenario corpus, \cref{ScenarioCorpus}, for an easier overview. Some Requirements can be placed in more than one domain, but for now, they will only be listed once.
\begin{itemize}
  \item[Planning]
    \item It must be possible to use leftovers.
    \item Base recommended recipes on what is already at home, lessening the need of shopping.
    \item Must be possible to plan for a different numbers of days.
    \item Being more than one person to plan, and especially if the planning is not detailed enough,it is easier to buy wrong items, or to forget to buy items, but it can help to improve the time it takes to shop, and lessen the amount of impulsive shopping. Furthermore, by having a long time to plan, there will also be more time to look through recipes and compare with the inventory.
    \item Certain people have to or chose to follow specific diets. Taking this into consideration, will expand the target audience furthermore.
    \item Another aspect that can be taken into consideration, is that people for different reasons could want to meet with other people. Therefore a feature to help gather people for social meetings, could be a possibility.
  
  \item[Shopping]
    \item Users will need to be able to plan meals, while they are out shopping.  
    \item Some users are more willing than others, to go to multiple stores while shopping.
    \item It varies when in the day, and how many times in a week people are willing to shop.
    \item Planned shopping lists reduces the shopping time for some users.
    \item Different aspects, while shopping can change what users what to buy, these aspects can be sales, impulsive shopping, or finding some interesting items that are on sale because they are close to the best before date.
    \item Not having a shopping list tends to result in buying to much, because it is not sure what should be bought, also people tend to do more impulse shopping, and having to go for more trips.
    \item Unfamiliar stores, where the user is not sure what can be bought in the specific shops, as well as long shopping list can result in a greater risk of having to visit more stores.
    \item Depending on the distance between home and stores, different requirements can be set, because a long distance can result in having to shop bigger quantities and it is more important not to forget to buy items.
    \item While shopping, one hand must be free to check the shopping list.
    
  \item[Cooking]
    \item Cooking can result in greasy hands, and if it is necessary to navigate in the program at the same time, it would have to be easy to do this navigation.
    
  \item[Inventory]
    \item with the inventory list in the program, it can be followed on the fly, and thereby the shopping list can be updated when the meal plan is changed.
    
  \item[General]
    \item Quality, organic, exiting, healthy and varied meals are important factors to users.
    \item Buttons in a program can be interpret in different ways, it is therefore important to make sure that the right icons are used.
    \item The more simple instructions are to follow, the wider a target audience can be.
    \item By having automated processes, the program will become easier to use, and thereby also widen the target audience.
    \item If the same storage is to be used on multiple devises, it would require synchronization between the devises.
\end{itemize}
  
In the following there are listed actions/requirements which happen in and out out the home, they are not sorted into domains because they are based on actions, actions/requirements to handle are:
  \begin{itemize}
    \item View list of stored items
    \item Update the stored item list, with used ingredients or ingredients that get old.
    \item clearing the whole stored items list. Could be that many items go bad, during to a holiday.
    \item Create a list of meals, that can be set to specific dates.
    \item Search through a list of recipes.
    \item Add or modify recipes.
    \item A list of items to ignore.
    \item While shopping, it should be possible to trace the status of the shopping list while shopping.
    \item If the meal plan is changed, the shopping list should change accordingly.
    \item A 30 day meal plan should not need to be bought the same day, therefore it should be possible to define the number of days to shop for.    
  \end{itemize}
  
  
The program will also have to use different technologies, these are described in the technology part of the PACT analysis, \cref{PACTech} are as follows;
  \begin{itemize}
    \item Touch screen
    \item Camera
    \item Microphone
    \item speakers
    \item Wi-Fi
    \item Telephone network
    \item Database
  \end{itemize}