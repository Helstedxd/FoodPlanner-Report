\section{Design Requirements}

When analysing the interviews and the scenarios, some design requirements will be discovered. In this section these requirements will be presented and described, which will provide a base for the design of the interface.

Not planning far ahead is one of the requirements that need to be considered in the design. This would require a easy way to distinguish between days, and only needing to select 1 day at a time.
Some people though, liked to plan more days at the same time, so keeping the days easily accessible from on another would be a way to fulfill this requirement.

Something that was found as a problem, was that some interview participants didn't consider what groceries they had at home when they went grocery shopping. This means that a requirement would be the ability to keep track of your inventory when shopping. The users should not only be able to see the shopping list, but should also easily access the inventory list.

When cooking, the quantity of the meal should be easy to edit, so both the shopping list, recipe and ingredients list will be updated. This is a requirement as the users like to be able to change the quantity depending on how many people are eating, or if they want to be able to make extra for lunch the day after.

To easily access a way of changing the inventory is an essential requirement, as the users might cook some food that are not planned in the program for various reasons. So being able to update the inventory should be easy to do.

