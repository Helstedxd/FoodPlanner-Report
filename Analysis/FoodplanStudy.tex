\section{Foodplan study (Stop Spild Af Mad)}
\textit{Stop Spild Af Mad} has conducted a study where 8 informants had had to create and follow a weekly food-schedule.

\subsection{Foodplan}
Each participants had to create a foodplan for the intervention week. They could freely choose the different recipes with inspiration from for example cookbooks or the internet.

\subsection{Planning}
The informants had some trouble with pulling them self together and getting started creating the foodplan, but they were happy to have it once it was done. It could be beneficial for the person that were in charge of grocery-shopping to be the one creating the schedule, since the involvement of other persons from the family could increase the time of making it.

\subsection{Moduels}
It was almost impossible for the informants to follow the schedule for a whole week, without having to reorganize it. Most of the informants had instead made small modules, that could be moved around depending on the family's situation. It was optimal with modules spanning 2-3 days. Varied meals was preferred, and not simply reheated meals.

\subsection{blabla}
The informants wanted a more realistic food-schedule as help, instead of having to construct it themselves from cookbooks and the internet.

\subsection{Less grocery-shopping}
7 out of the 8 participants was shopping groceries almost every day. This was not seen as a nuisance, but after having planned grocery-shopping for a whole week, they experienced a big relief in their everyday life.

Plus: Timesaving, Freedom, less buying on impulse (saving money).

%Source: http://www.stopspildafmad.dk/madspildsrapport.pdf

\subsection{our solution stuff }
The participant had trouble getting started creating a new food plan for the week. They also wanted a more realistic food-scheduele instead of having to create it from cookbooks.
A solution to this problem might be solved by automating a lot of the tasks involved with having to come up with the different recipes and do the grocery shopping.
By having an application come up with suggested meals based on different aspects such as:
- the food you already have
- the food you like
- the price
- the stock of stores nearby
- and more?
people could quickly create the schedule or maybe even have the application create it for you.

