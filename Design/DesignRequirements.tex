\section{Design Requirements}

When analysing the interviews and the scenarios, some design requirements will be discovered. In this section these requirements will be presented and described, which will provide a base for the design of the conceptual design process.

Planning a varying number of days is one of the requirements that need to be considered in the design. This would require a easy way to distinguish between days, and only needing to select 1 day at a time.\fxnote{"Planning a varying number of days " Det er ikke tdeligt nok hvordan dette skal forstået. Jeg vi forslå at man skriver noget a la; When a auser tries to schedule melas for their timeplan, some might be satisfied by just being able to plan two or three days head....}
Some people though, liked to plan more days in the same session, so keeping the days easily accessible from on another would be a way to fulfill this requirement.

Something that was found as a problem, was that some interview participants did not consider what groceries they had at home when they went grocery shopping. This means that a requirement would be the ability to keep track of your inventory when shopping. The users should not only be able to see the shopping list, but should also easily access the inventory list\ref{Interview}.

When cooking, the amount of people the meal will be prepared for should be easy to edit, so both the shopping list, scheduled meal and ingredients list will be updated. This is a requirement as the users like to be able to change the quantity depending on how many people are eating, or if they want to be able to make extra for lunch the day after. %Sikre at de lister der bliver nænt har samme navn som i programmet/klassediagram!

To access a way of changing the inventory should be simple task, and is an essential requirement, as the users might cook some food that are not planned in the program for various reasons. So being able to update the inventory should be easy to do.