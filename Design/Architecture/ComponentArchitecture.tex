\section{Component Architecture}
In this section a class diagram will be designed to show the specifications of all complex components. This is done in order to create a flexible and understandable structure of the system.

%Insert classdiagram, Spørg Anders hvis det endnu ikke er inde
\label{LayerDiagram}

\Cref{LayerDiagram} shows the class diagram for the components of the system. The pattern that have been used is the MVVM pattern which "Forklaring om MVVM, se det som Mathias har lavet".

\textbf{View}
The view component spans over the pages(windows) that are used to interact with the user. The responsibility of the View lies in that it should send information to the View Model such that the user can interact with the system.

\textbf{View Model}
The view Model component spans over the classes that are associated with the pages in the view. The responsibility of the View Model is to give functionality to the model. The component uses the classes found in the Model. The view Model also contains an Navigator which is used by the View and View Model itself and is responsible for letting the user navigate trough the different pages.

\textbf{Model}
All the classes that are used in the system is located in the Model component and is used directly by the View Model. The responsibility of the Model is to provide classes that can be used to manipulate data and synchronise such data with an database. The objects that can be created through the Mode is used in the View Model. Another part of the Model is the settings. These settings are saved locally on the machine and is responsible for allowing some personal customization of the local processor. 

There are an external standard component which is not shown on the diagram, the component is the Entity Framework which is used to simplify the data modification when it is saved or retrieved from/to the database. \Cref{EntityFramewotk} describes the framework in more detail.