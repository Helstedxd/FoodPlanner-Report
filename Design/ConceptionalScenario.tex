\subsection{Conceptual Scenarios} \label{conceptualScenarios}
In this section conceptual scenarios are defined through an abstract view on the stories in \cref{UserStories}. The scenarios consist of generalisations of situations that are alike, and will together with the user stories be used in the following understanding process.

\subsubsection{Conceptual scenario 1}
People who plan to make a meal look in their storage, freezer and cabins, to see if meals can be created from what is already at home. If the shopper have not created a shopping list, he can get confused about what really is needed, thereby buying to large quantities, furthermore they easily get tempted to do impulsive shopping. Lastly, if there is not created a shopping list, shopping will have to be done more times during a week, because items can be forgotten and because it is harder to plan a longer period in advance, when you have to remember all you must buy.

\subsubsection{Conceptual scenario 2}
If more than one person is making a shopping list, and this shopping list is not detailed enough, it can result in getting wrong items or missing to buy certain items. By having this shopping list to follow, users are less tempted to do impulsive shopping.

\subsubsection{Conceptual scenario 3}
By planning long time in advance, a shopping list can be created over a longer period of time, where items are added, when the person who is writing the shopping list, remembers what is needed. This also allows for time to go through recipes and compare these to the inventory of the household, this minimises the need for items to bought, and thereby helps to minimise food waste. Lastly, as with scenario B, having a shopping list reduces the amount of impulsive shopping.

\subsubsection{Conceptual scenario 4}
When shopping in an unfamiliar store, the person doing the shopping will be less certain of what actually can be bought in the given store, therefore the likelihood of having to go to multiple stores are greater.