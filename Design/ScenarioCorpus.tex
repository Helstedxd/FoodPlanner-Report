\subsection{Scenario Corpus} \label{ScenarioCorpus}

In this section scenario corpus' are going to be described. The scenario corpus' will be based on the interviews in \cref{InterviewAnalysis}, conceptual scenarios in \cref{conceptualScenarios}, the PACT analysis in \cref{PACTAnalysis},the personas, which can be found in \cref{PersonasAppendix}, and lastly the design requirements listed in \cref{DesignRequirements}. To make the scenario corpus', similarities and differences of these sections and chapters are found, each scenario will cover a domain of requirements and activities, that are similar. The domains are as follows; planning, shopping, cooking, inventory and general. Since some of the domains a quite small they will be merged into fewer scenarios

\subsubsection{The first scenario}
Jane lives with her boyfriend Jacob, and is the one to normally do the planning, it is Sunday, and she decides to plan the meals for the next weak, while Jacob is out for football training. Furthermore have they just started a new diet, where they try to avoid certain foods.
\begin{enumerate}
  \item Jane sits down to start the application to help her plan the next week of meals. Starting from today she looks at recipes and sees there is a recipe when they only need to get carrots. Jane texts her boyfriend to tell him to buy some carrots for tonight's meal, she then contentious the planning
  \item Thinking that carrots are normally sold in big bags, Jane schedules four of the meals in the coming week, to have recipes where there are used a lot of carrot.
  \item Jane plans the last three meals to be placed in between the four meals with carrot, so they would not have to eat to much carrot.
  \item Jane and Jacob are having some friends over the next Saturday, therefor Jane also plans this meal for five persons instead of the two, that they normally are.
  \item Later Jacob comes home, and he remembered to buy carrots. Unfortunately for Jane's planning he only bough a small box of carrots, because he had read in the text message that it was carrots for tonight's meal, and he therefor did not want to end up with to many carrots.
\end{enumerate}

\subsubsection{The secon scenario}
It is now the day after and Jane is going to do some shopping after work, she knows that there are sales in two stores, placed close to each other, so she drives there, to do the shopping
\begin{enumerate}
  \item When she arrives at the first store she opens her application with the shopping list, she does not want to shop for the entire week, because she knows she will have to buy a lot. So she sets the shopping list to show what needs to be bought for the next three days.
  \item She sees that she has to buy a lot of carrots and remembers that she had scheduled four meal with carrots, because she thought her boyfriend would have bough a large amount the day before. But she does not want to have four days with a lot of carrots, if she can avoid it, so she reschedules two of the meals, before she contentious the shopping.
  \item Jane walks past the shelves with candy when she sees there is a sale on chocolate, she would like some for the evening, but is also quite certain that they have some at home, so she opens the inventory list, only to see that they barely have any chocolate left, therefore it goes into the basket as well.
  \item Jane knows what there is on sale in the other shop, and therefore she does nit buy it in this store, also there are some items which she cannot find, she therefore proceeds through the check out and crosses off the bought items, before going to the next store and getting the rest.
\end{enumerate}

\subsubsection{The third scenario}
Jane is now home from the shopping, and it is time to organize the inventory and begin cooking.
\begin{enumerate}
  \item Jane gets home and begin to put the bought items away, when she gets to the chocolate, which she bought even though it was not on the shopping list, she remembers that she has to manually add this to their inventory, because it has not been automatically added from the shopping list.
  \item After putting away all the items, she begins to cook, Jane has never cooked this meal before, and therefore rely heavily on the recipe. she starts by finding the needed ingredients from a list which she can easily scroll through with one finger.
  \item after finding all the ingredients she begins to cook, the recipe requires Jane to use her hands, which makes them greasy, furthermore when she has to check up on the amount she needs of certain items, and when she has to go further into the preparation steps, she has to scroll, but with her greasy hands she must to this with er elbow, which she can, because the touch area for scrolling is quite large. 
\end{enumerate}