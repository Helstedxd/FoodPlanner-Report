\section{Concrete scenarios}
\textbf{*intro*}

\paragraph{MealPlan/01}
In this scenario a user wants to plan three meals for three days. The person does this by choosing a day and then a recipe three times. The person does this at home as she usually does it there.

Title: Three day planning via days
Scenario type: Concrete scenario
Overview: 
	People = Peter Nielsen, average smartphone user. Works at an office
	Activities = Planning meals
	Context = In his home. More particularly in his living room.
	Technology = His smartphone.
	
Rationale

In this scenario a person wants to plan three days while sitting at home. He will use his smartphone's touch screen when planning. The scenario focuses on the steps he takes to plan the three days.
	A1. It is 10:30 and peter would usually be at work this time a day but today is Saturday. He just ate breakfast and remembered that he had to plan what he was gonna eat the three next days. He knows that he feels like eating steaks one of the days but the two other days is not decided yet.
	A2. He leaves the comfy sofa to get his smartphone from the charging station and walks back to the sofa. He starts searching for the program on his smartphone and starts it when he finds it.
	A3. On the start screen of the app he briefly views the different options[1]. He chooses to touch the view meal plan icon[2] and gets directed to an overview of his meal plan[3]. 
	A4. on the next screen he gets a display of the next few days that he can plan. He starts by touching the next day and gets a detailed view of what meal is planned[4]. Currently there are no meals planned so he finds the "add meal" buttons[5] and gets directed to a new window[6]. 
	A5. This window shows him an overview of different categorized recipes[7]. He starts browsing through the recipes until he finds a steak recipe[8]. 
	A5. He is interested in this recipe and touches the picture on the screen[9]. He then gets an overview of the recipe including a description, ingredients and how to guide[10]. He then decides to add this recipe to the first day in his meal plan by touching the add to meal plan button[11].
	A6. He now gets an updated view of his meal plan with the recently added recipe on the first day[12]. He is satisfied with his choice and now moves on to the next day. 
	A7. He now follows the same procedure for the rest of the days that he wants to plan. When he has planned all the days he exists the program and walks into his study room to do some work.
	
\paragraph{Notes to scenario MealPlan/01}
1 What options is there and how many? it is important to consider to avoid screen clutter.
2 How are these icons designed? icon/symbol, colour theme etc.
3 It should be considered what information to show e.g. number of days, day information and the visual display etc.
4 what information is important for the user to see here? meal titel, ingredients, quantity etc.
5 how dos this button look? is it even a button or a swipe bar.
6 should it jump between different windows or something else? (this can be critized elsewhere)
7 what categories and how are the criterias they have?
8 how does the browsing work (scrolling, buttons)?
9 should the picture be pressed or dragged or something else?
10 what information should be displayed and how is it arranged on the screen?
11 same consideration as note 9
12 is there a refresh button or does it update on the fly?

\paragraph{MealPlan/02}
Title: One day planning via recipes
Scenario type: concrete scenario
Overview:
	People = Maria Johnson, average smartphone user, Studies economics at a university. 
	Activities = Planning a meal.
	Context = on the way home from the university.
	Technology = Smartphone.

Rationale

In this scenario a person wants to plan a meal by choosing a recipe that is interesting. The person does this by choosing a recipe and then adding it t a day. The person does this from a location outside of the home. 
	B1. It has been a long and stress full day for Maria at the university. She wanted to plan her meal at lunch but did not find any time to do it. She is now on her way home in the bus and finally feels that she can figure out what she wants to eat tonight.
	B2. She opens her purse and pulls up her smartphone. She quickly checks her e-mail and afterwards finds the meal planning program and starts it. 
	B3. On the start screen she is presented with different icons. She touches the recipe icon and a new window appears.
	B4. The screen now shows her many different recipes and options to browse through. She starts to browse them looking for what she wants to eat. 
	B5. She finds a recipe that she is interested in and touches the icon for that recipe. A more detailed overview of the recipe appears and she quickly scrolls through the overview.
	B6. After reading about the recipe she decides not to have it anyway. She now goes back to the recipe browsing screen and continues her search.
	B7. She now finds a new recipe that she wants to view. She touches the recipe picture and gets an overview of that recipe. 
	B8. She decides that this is the recipe that she wants. She now touches the add to meal plan button and gets some options to associate the recipe with a specific day.
	B9. She is now satisfied with her choice and closes the program and start wondering what home reading she has to do for tomorrow.
	
\paragraph{Notes to scenario MealPlan/02}   