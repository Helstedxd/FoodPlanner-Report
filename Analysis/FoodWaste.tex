\section{Food waste}

Every year danish households throws away nearly 237.000 tons of food, that could have been eaten by people. This is estimated to be nearly 16 mio DKK. About 20 percent of a danish households foodbudget is wasted by throwing away food that could have been eaten, this means that if there was no food waste at all, 1 million people could be fed, this is nearly 1/5 of the danish people.

The people who throws away most food, are the people living alone. The average person living alone throws away 98.8 Kg of food per year, where as a household with 5 people throws away 46.8 Kg of food per person per year. It is mostly Dairy, vegetables and bread that get's trown away, though in december a lot of meat is also thrown away, due to the fact that it is christmas, so people eat more food.

Since 2007 the danish people have been purchinsing 10 % less food, a lot of this food used to end up in the trash can as food waste. 54 % of the danish people rarely or never uses the food from the night before for later eating, and 69 % does not use left overs for lunch the day after. Elderly people are better at using leftovers than younger people.

20 % of danish people does not make a grocery shopping list before going grocery shopping. 52 % of people does not make a meal plan, 31 % does sometimes and 17 % does it often.

81 % of danish people would use leftovers to avoid food wate, 50 percent would make a mealplan and make a grocery shopping list. 59 % of the consumers believe that the danish food waste is their own fault.
