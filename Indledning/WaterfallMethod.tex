\section{Choosing and explaining the use of our chosen structure method}

In this section we will describe the Waterfall method and the Iterative method which both are used to structure a project. We will discuss the concepts of them and their strengths and weaknesses. This information will be used to support our choice of method. The chosen method and its application in the report will be described afterwards.

The waterfall method iterates the project process one time and avoids revisioning parts that have already been finished. That means that a part is worked on until it satisfies certain requirements and then the project members move on to the next part. The iterative method focuses on multiple iterations of the project process and in so doing the project many times, expanding the product and report each time.

Both methods have their strengths and weaknesses and one method might have a strength that is a weakness for the opposite method. One of the strengths of the Waterfall method is the consecutive order of progress the structure gives. This gives the project members an overview of how far they are in a specific part or in the overall progress of the project because they can say when a part is done. Another strength of this method is the ease of setting deadlines for the parts in the report and using these deadlines to measure the progress of the project. The consecutive nature of the method can also become a weakness. During the project information can change and this can result in previous parts of the report not including or reflect this new information. This also leads to another weakness which is simply the question, is the product of the project really complete when the project is done. The project can expand and new information can change, add or subtract requirements for the final product. So it is hard to say that the product is complete when the project is done so it is important to keep this in mind when consecutively progressing through the project process. The other method that we have mentioned is the iterative method. This method has the strength that is dynamic. Each iteration of the project process allows the addition of new information. This means that this method adapts very well to new information and changes. When using this method it can be difficult to keep an overview because each iteration adds more information and gives a better understanding of the project context, development and requirements of these. This could lead to an endless circle of improving and expanding the project. So when is the project actually done when using the iterative method. It can be very difficult to say whereas the Waterfall method gives a structure that restricts the multiple iterations and sets a end point. 

The Waterfall model gives a good structure with an easy overview of progress. It is also a good choice because we can define what system requirements and goals we have after each part of the report and use these in the following parts and at the end know that we have completed the goals that we have set. Therefore we will be using the Waterfall method in this project.       

Writing a report using the waterfall method as a structure is done by writing five different parts which covers the project process from start to finish.

\textbf{The analysis part} focuses on getting to know The project in context e.g. to the audience or the environment. This is the part of the report that is called problem analysis.It is based on relevant sources e.g. articles, reports, questionnaires or interviews which satisfies a certain standard of source criticism and relevance. In our report the information that was gathered through external sources and interviews was used to define specific goals and requirements for the following parts of the report.

\textbf{The design part} is used to identify design specifications that is needed in a certain product. In our report, the design part focused on the principles learned in the courses "Design and Evaluation of User Interfaces" and "Object Oriented Analysis and Design". In this part of our report the interface design was modeled by using scenario- based design and the architectural design was created by using Object Oriented Analysis and Design techniques. 

\textbf{The implementation part} Is the part in which the software is created. The design specifications is used to make the design, and the needed functionality requirements which was identified through the analysis part is implemented. The outcome of this process was a working program.

\textbf{The system test} in his part the system is tested to see if it satisfies the stated requirements and goals in regards to both usability and quality. In our report we tested the system to secure that it was of high quality and we also tested the usability of the program.\fxnote{skriv mere præcist nå testing er done} 

\textbf{The system in operation} Is concerned with describing the system when and after it is published. Our report incorporates this part as a reflection and evaluation that is written as a closure to the report\fxnote{beskriv bedre når det er lavet}. 


%The reason that the waterfall method was chosen for this project, was that a lot of new information would not come throughout the project. Food waste is and will be a problem, and some sort of product is needed to change this. Therefore it seemed natural to gather all the information at first, and work out from that. The waterfall method lived up to the expectiations of the group, meaning that deadlines where kept so the project was on track all of the time.