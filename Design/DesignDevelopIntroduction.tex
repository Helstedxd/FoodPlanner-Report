This section will use the physical design process, in which the \nameref{Sketches} in \cref{Sketches} and the \nameref{ConsMod} in \cref{ConsMod} are combined, to produce the design language for the program. The physical design process is used to define the look and feel of the program. The process also defines the allocation of functions and the knowledge between a user and a device. The product, which is the design language, is a standardisation of the interaction with the program and the physical style of the program e.g. buttons, shapes and colours. This means that the design language is a guideline during the development phase and therefore, that the actual program can differentiate itself from the design language.