\chapter{Problem Statement}
It is a problem that so much food is going to waste. This is a consequence of bad planning when people are shopping for their week to come, and because the consumers wants fresh groceries. Miscalculations are bound to happen if the the one doing the shopping is part of a bigger group, such as a family. Shopping is also time consuming if people want their groceries at a low price. It can be hard to balance the price and time if the cheap stores are far away, or if the consumer is unaware of recent discounts. It becomes extra time consuming if the shoppers have special food diets or preferences. Users who live alone also find that they have a big foodwaste, and have a higher cost per meal than users who live and eat together with others.

\textbf{How is it possible to create a software solution that caters to the needs of those who wants to save resources, but still maintains a low foodwaste level, and still wants to be able to fulfill their food lifestyle choices.}

%How can a software solution make it possible to plan ahead and also take into consideration any relevant kind of special requirements that the user has.