\subsubsection{Concrete scenarios evaluation} 
The following text will present the results found by evaluating the different scenario notes. The intentions with performing an evaluation of the scenario notes is to uncover design considerations that has to be addressed. The section is structured so that the domains identified in \cref{ScenarioCorpus}, will be used to categorize the different design considerations. If a domain does not have any associated design considerations it will not be used.  

\emph{Planning}

This domain is associated with design considerations from concrete scenario 1, 2, 4 and 6. The considerations are as follows:

\begin{enumerate}
	\item Enabling the user to specify and find the right recipe.	
	\item Indication the number of ingredients the user has for a meal.
	\item How is a meal planned e.g. number of steps.
	\item Making the addition of planned meal ingredients to the shopping list automated by the program.
	\item How does the user add a meal to a day on his meal plan e.g. using a button or dragging.
	\item How does the user search for a specific recipe?
	\item how many criteria or search options does the user have when searching for recipes?
\end{enumerate}

\emph{Shopping}

This domain is associated with design considerations from concrete scenario 1, 2, 3, 4 and 5. The considerations are as follows:

\begin{enumerate}
	\item Can the user how many days to shop for?
	\item How does the user "buy" some ingredients on the shopping list but not all?
	\item What information is shown for each ingredient on the shopping list and how is it shown?
	\item Should the user cross off items on the shopping list or how is it indicated that an ingredient is bought?
	\item Can the user add items to the shopping list?
	\item What information should the user input when adding an ingredient to the shopping list?
	\item Can multiple ingredients be added to the shopping list at once?   
\end{enumerate}

\emph{Inventory}

This domain is associated with design considerations from concrete scenario 1, 2 and 3. The considerations are as follows:

\begin{enumerate}
	\item What information is shown to each ingredient on the inventory list and how is it presented? 
	\item How is this list kept up to date? can the user add items to the inventory list?
\end{enumerate}
