\section{Scenario Corpus} \label{ScenarioCorpus}

In this section the scenario corpus is going to be described. The scenario corpus will be based on the interviews in \cref{InterviewAnalysis}, conceptual scenarios in \cref{conceptualScenarios}, the PACT analysis in \cref{PACTAnalysis},the personas, which can be found in \cref{PersonasAppendix}, and lastly the design requirements listed in \cref{DesignRequirements}. To make the scenario corpus', the similarities and differences of these sections and chapters are found, each scenario will cover a domain of requirements and activities, that are similar. The domains are as follows; planning, shopping, cooking,inventory and general. Each domain will be described and a scenario will then be written

\subsection{Planning}
A big part of lessening food waste is to plan ahead, furthermore planning can save time, by ensuring that everything needed is bought at the first shopping trip. 

\subsection{Grocery shopping}

Based on the user stories in section \ref{UserStories}, grocery shopping is most often done when people already are out of the house. So having a food inventory administration system and keeping a shopping list, makes people able to shop whenever they find it convenient. Some people might go home before shopping to check what they already have, but this is not needed if it is easy to see what you have by looking it up at a mobile application.

\subsection{Cooking}

If cooking is done at a specific time, planning meals is also most optimal like the persons that behave like the conceptional scenario B \ref{ConceptionalScenarioB}, because you can go shopping a couple of days before you need the food, and do not have to shop every day, so if the person gets off work late one day, the shopping could have been done some days before, so the cooking will still be done, though the shopping does not need to be done at the same day.

\subsection{Food inventory administration}

In \ref{ConceptionalScenarioA} the people do have a feeling of what is being needed in the home, where as the people of \ref{ConceptionalScenarioB} plan their meals from what is in the home. This is a similarity since if you have a feeling of what you have to buy, you have a feeling of what you already have. When the system keeps track of this there is no need to have a feeling, and the risk of buying something already owned will be reduced. If the system has a design that is easy to browse through, and find what is already owned, the amount of items that will be bought, but already are owned will decrease.