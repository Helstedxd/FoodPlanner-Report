\section{Future Improvements}
In this section the improvements which could be made in future will be listed and described. An reflection on how to program could be used in other contexts will also be made in order to explore new areas.

\subsection{Improvements}
The improvements mentioned in the \nameref{DelimitationSection} should all be included in a full version of the system. These changes should be accompanied of those changes presented \nameref{UsabilityTestSection}.

More changes have been thought of while developing the system and writing the documentation. These changes or improvements will be described this subsection.

An improvement to te system would be to include an budget page where the user could view his or her spendings for the month. It should be possible to allocate an budget for the month, and the program should then generate an meal plan that will fit to the budget. The shopping list should display the total cost of the items present on the list. This change would be beneficial as some user have applications on their smartphones or computers that help them with their budget. By having a part of the user budget in the program, it becomes more of a complete package for the user.

Something else that was thought of would be the possibility to pick or exclude shops to shop from. When the user is browsing their shopping list, they should be able to get recommendation on where to buy the different product. Some of the shops might lie on the users route back from home or their children's school. This feature could be expanded to allow the user to set an maximum distance of how far they would like to travel when shopping.
køb i en forret men udeluk andre

sart shopping priser - køeb emen hente forskllige tidspunkter
 

%If more time would be assigned to the project, the previously mentioned improvements will be implemented together with this list of features:
%\begin{itemize}
%\item 
%\end{itemize}

\subsection{Reflection}
The program could be changed so the focus would lie on bigger groups of people instead of families. A kindergarten could implement the system in order to keep track on what the kids should have for lunch. Some of the children could suffer from milk allergy and therefore needs special diets in order to keep milk out of there food. Other kids could be diabetics and so. The system should allow the pedagogues to make appropriate servings of lunch to the kids, while helping them come up with recipes for alternative dishes. This would make the administrative work easier and keep the food waste level down at a place with many individuals. In order to make such a system new recipes have to be implemented as the current system is focused on dinner recipes and not lunch recipes. The recipes should also be focused on children and exclude gourmet and other inappropriate recipes.

The program could also be redesigned to be website for a catering company. The customer would be people who would order food to their parties. They would do so by going to the website and then choose the dished they would like to order together with the expected guest count. The catering company would then get an order with a list of recipes they would have to cook together with an auto generated shopping list and price. The price would be based on the shopping list and other factors such as how time constrained the dishes would be to cook or alike. This could double up as being an administrative program that shows the appointments they would have with their customers. This would hep the company save time by automatically making shopping lists and scheduling there calender. The food waste aspect would not be as big an aspect as the amount of food made would probably not be changed by using the system. However, it might help the company to not buy unwanted ingredients as the system would keep track of what there is currently stocked. The obstacles that would arise if this system were to be made would be to convert the entire program to be a website instead, even though some of the current features would not be ported. The interface and how to navigate it would need to be changed entirely as the there would be a customer view and administrative view for the staff of the company. 