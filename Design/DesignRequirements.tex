\section{Design Requirements}
When analyzing the interviews in \cref{InterviewAnalysis}, conceptual scenarios in \cref{conceptualScenarios} as well as the PACT analysis in \cref{PACTAnalysis}, lastly also personas, which can be found in \cref{PersonasAppendix}, some design requirements will be discovered. In this section these requirements will be presented and described. And used with the rest of this chapter to define a scenario corpus

\begin{itemize}
  \item It must be possible to use leftovers.
  \item Quality, organic, exiting, healthy and varied meals are important factors to users.
  \item Must be possible to plan for a different numbers of days.
  \item Users will need to be able to plan meals, while they are out shopping.
  \item Short shopping times are preferred by many users.
  \item Some users are more willing than others, to go to multiple stores while shopping.
  \item It varies when in the day, and how many times in a week people are willing to shop.
  \item Planned shopping lists reduces the shopping time for some users.
  \item Different aspects, while shopping can change what users what to buy, these aspects can be sales, impulsive shopping, or finding some interesting items that are on sale because they are close to the best before date.
  \item Not having a shopping list tends to result in buying to much, because it is not sure what should be bought, also people tend to do more impulse shopping, and having to go for more trips.
  \item Being more than one person to plan, and especially if the plan is not detailed enough,it is more easy to buy wrong items, or to forget to buy items, but it can help to improve the time it takes to shop, and lessen the amount of impulsive shopping. Furthermore, by having a long time to plan, there will also be more time to look through recipes and compare with the inventory.
  \item Unfamiliar stores, where the user is not sure what can be bought in the specific shops, as well as long shopping list can result in a greater risk of having to visit more stores.
  \item Depending on the distance between home and stores, different requirements can be set, because a long distance can result in having to shop bigger quantities and it is more important not to forget to buy items.
  \item Buttons in a program can be interpret in different ways, it is therefore important to make sure that the right icons are used.
\end{itemize}