\section{Avoiding Food Waste} \label{HowToAvoidFoodWaste}
20 \% of groceries bought by a Danish household is thrown away, making it essential to find ways to decrease or prevent food waste. When throwing away food, it is not only groceries that could have been eaten, which are going to waste, but also the energy used to grow, make or harvest the product.

According to The Danish Ministry of the Environment\cite{madSpild_Notat}, one of the initiatives that can be taken against food waste, is to make recipes which incorporate the leftovers and other uncooked groceries. By doing this, food that might otherwise go bad, will be used instead of being thrown away.

Another initiative that can be taken, according to The Danish Ministry of the Environment, is to sell quantity discount, with the option to receive some of the groceries later. An example would be if one cucumber is sold for 10 DKK, while two are sold for 15 DKK, some people would choose to buy two cucumbers because of the quantity discount, even if they just needed one at the moment. With a "recieve later" option, the consumer would be able to buy two cucumbers, but only receive one at the moment. Then the consumer would be able to come back to the store another time to get the other cucumber. An initiative like this have already been made by the English supermarket chain Tesco. The initiative is called "Buy One Get One Free Later", and is an initiative for fresh groceries like fruit and vegetables.

To give the consumer better information about when food is going bad is also an initiative that The Danish Ministry of Environment is recommending. By informing consumers of what "Best before", "Last day of sale" and so on means, would get the consumer to use the food more properly. Also by getting the consumer to rely more on smell, taste and looks, rather than just the date of production on the groceries, would lead to less food waste.

Only using the groceries necessary is also a way to reduce food waste. Following the steps of a recipe will help, if the recipe is specific. If for example the recipe is for four persons, and four persons will be eating, and the recipe is correct, food will not be wasted. Even if there are leftovers after the meal, eating them or using them as lunch the next day, will reduce food waste\cite{madSpild_MindreMadspild}.

Keeping track of products that is being thrown out is another way to reduce food waste. By detecting patterns it can be discovered if the consumer may be overbuying food. An example is if 500 grams of beef is thrown out each week, the consumer may be overbuying beef\cite{madSpild_Greatist}.

Eating the food which is closest to the expiration date is a good way to prevent food waste as well. If the refrigerator is being filled up as the week goes by, new products might be put in the front as they are bought last. By keeping track of the expiration dates by either organizing the products and having the oldest in the front of the refrigerator, or by writing down the expiration date of the different products, food waste could be reduced, as old food gets eaten first.

According to United States Environmental Protection Agency\cite{madSpild_EPA}, being creative with leftovers is another way of preventing food waste. For example if a loaf of bread has gone stale, cutting it up and using it as croûtons is a great way to reuse it, instead of just throwing it away.

In conclusion it is essential to find solutions to minimize food waste. In this section, suggestions have been made, that could help reduce food waste, and thereby help preserving the environment as well.