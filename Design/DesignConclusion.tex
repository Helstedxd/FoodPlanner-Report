\chapter{Design Conclusion}
In \nameref{ScenarioBasedDesign} \cref{ScenarioBasedDesign}, the interaction design of the program was defined. This was done by using the Scenario-based Model from the \textit{Design and Evaluation of User Interfaces} course. This model used real world experiences of users as \nameref{UserStories} written in \cref{UserStories} and from these \nameref{conceptualScenarios} \cref{conceptualScenarios} and \nameref{ConcreteScenarios} \cref{ConcreteScenarios} was created. The concrete scenarios was use to develop \nameref{Sketches} \cref{Sketches}. The conceptual scenarios and Stories was processed to produce \nameref{DesignRequirements} \cref{DesignRequirements} and \nameref{ScenarioCorpus} \cref{ScenarioCorpus} which was then used to define a \nameref{ConsMod} \cref{ConsMod}. The Paper prototypes and the conceptual model were then combined to formulate the \nameref{DesignLanguage} \cref{DesignLanguage} for the program. The design language defined a standard for the interaction design of the program.  

In \nameref{Architecture} \cref{Architecture} the \nameref{ComponentArchitecture} \cref{ComponentArchitecture} defined the architecture as a layer pattern (MVVM). The \nameref{ProcessArchitecture} \cref{ProcessArchitecture} defined the process architecture, by finding the distribution pattern which suited the system best and then adapting that pattern to the actual system. The distribution pattern became a variant of the centralized pattern. Both the component architecture and the process architecture was supported with basic knowledge by \cref{DataBaseModel/Classes}, \cref{MVVMSection} and \cref{EntityFrameworkSection}.   
