These personas are generated from user stories, in \cref{UserStories} and conceptual scenarios, in \cref{conceptualScenarios}, furthermore the personas will be used as a helping tool for creating concrete scenarios and to help set requirements and a scenario corpus.

\subsubsection{Henrik Jensen} \label{PersonaHenrikJensen}
\begin{figure}[H]
	\includegraphics[width=0.30\textwidth]{Grafik/FoodPlanner/PersonaHenrikJensen}
\end{figure}
\begin{itemize}
	\item Age: 38
	\item Relational status: Single
	\item Occupation: Working as an IT consultant.
	\item Preferences: On a busy schedule Henrik rarely plan what to eat, until he on the way home drives by a store.
\end{itemize}
Henrik drives home after he has finished a meeting with a costumer. The meeting dragged on and Henrik just want to get home and get a nice dinner.

Henrik parks at the local store, where he starts the application. Thinking he want to cook something with chicken, he searches for recipes containing chicken, and finds 'Jamaican curry chicken', he add the recipe to his meal plan, and entering the store he navigates to the shopping list, containing the needed ingredients, which the application has not listed under storage.

Just as he enters the store he changes his mind on what to eat, because he sees some really cheap beef. He now searched for recipes with beef, and decides on 'swedish cocktail meatballs', next Henrik updates his meal plan thereby the shopping list has also changed.

When Henrik sees he has to buy eggs, he gets confused, because he is almost certain that he has eggs at home, but trusting the program he gets a pack of eggs.

When Henrik is home and begins to unpack, he notices that he actually did not have any eggs, and is therefore happy he trusted the program.


\subsubsection{Peter Nielsen}	\label{PersonaPeterNielsen}
\begin{figure}[H]
	\includegraphics[width=0.15\textwidth]{Grafik/FoodPlanner/PersonaPeterNielsen}
\end{figure}
\begin{itemize}
	\item Age: 23
	\item Relational status: With girlfriend
	\item Children: None
	\item Occupation: Student
	\item Preferences: Does not spend more time in the kitchen than needed.
\end{itemize}
Sitting in class, Peter knows that he should be the one to buy what is needed for dinner, but Peter is not the most imaginative when it comes to deciding this, so he texts his girlfriend, who at the time is not home either, she tells peter to just get what is needed for the recipe, where they already have the most ingredients. Perter looks up a list sorted by what they already have, with the most completed lists at the top, he finds 'Beef, Snap Pea and Asparagus Stir-Fry', and adds it to the meal plan for the evening

On his way home Peter goes to the closest grocery store, and shops the needed ingredients, and without neither him or his girlfriend needing to be home to see what was needed to be bought, they had what they needed to prepare the meal.

\subsubsection{Anne Madsen}
\begin{figure}[H]
	\includegraphics[width=0.25\textwidth]{Grafik/FoodPlanner/PersonaAnneMadsen}
\end{figure}
\begin{itemize}
	\item Age: 18
	\item Relational status: Single
	\item Children: None
	\item Occupation: Student
	\item Preferences: Likes to spend time in the kitchen to prepare a good fresh meal, as well as spending time on preparing what meals to have, days in advance.
\end{itemize}
Anne has the day off, from school, when she looks at her meal plan, to find that she has only planned for the next two days. Since she has the time she decides to plan the next two full weeks.

Using the application Anne quickly fills out all the days, and with more time to spend, she wants to do the shopping for the next four days. She therefore sets her shopping list to only include the next four days, then she goes to the city, to do the shopping.

While doing the shopping she notices that none of the stores she has been to, have two of the ingredients she needs for the meal on the fourth day. Knowing that she has to go to another part of the city the day after tomorrow, she sets the shopping list to only show three days, and then continues the shopping, only getting a few of the items needed for the fourth day.