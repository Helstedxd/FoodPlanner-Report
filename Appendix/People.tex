\subsubsection{People} \label{PeoplePACT}
In this analysis there will be focus on the social, physical and psychological differences between people and there will be a description of their different motives and preferences. The information used is gathered from the user stories in \cref{UserStories}, and from the information gathered in \cref{InformationGathering}, furthermore a description of this will be used to get an idea of who could benefit from a system, that would help them organize a household inventory, grocery shopping and preparing meals more efficiently.

\paragraph{Physical differences}
People have different physical abilities. Some people have bad vision, some have bad hearing, and so on, taking this into consideration is an important aspect of a PACT analysis.

There can be a great geographical distance between a person's home and grocery stores, People who have trouble traveling this far, will need to go shopping as rare as possible, this can be helped with good planning and a detailed shopping list.

Looking at the ergonomically aspects of the application\fxnote{Use another word than application?} is not necessary, since the ergonomics are decided by the company who creates the device that the application run on.

\paragraph{Psychological differences}
People vary in the way that they function psychologically. When taking psychological differences into account, it will be most ideal to take the weakest user into account, so the program can be used by as many people as possible. Some common traits to look at when designing software are:
\begin{itemize}
    \item The meaning of buttons
    \item Easy remembering of instructions
    \item Ease of use, through automated processes
\end{itemize}
The meaning of icons can vary, between cultures and countries, it is therefore very important to make sure that buttons will be perceived correctly. As a result form this, it cannot be certain that the program can be used in other countries than Denmark, which is where the program will be designed for.

It is also important that no instructions or commands are too long, because it will be hard to remember for some people, and when making a product you must take the weakest user into consideration when looking who to design the product for.

Automated processes reduce the amount of operations to remember, in order to do certain operations.

\paragraph{Social differences}
Social differences result in different requirements for a product. Below are mentioned different groups of people, whom have been divided into different categories.

\emph{People following a special diet, or having specific wishes for food plans,}
will have to buy food based on the diet, and sometimes prepare it in a specific way. Some examples of these people could be:
\begin{itemize}
\item Athletes (Bodybuilders)
\item Vegans/vegetarians
\item Organically minded
\end{itemize}

\emph{People who want to save time when planning, shopping, cooking and preparing the food,}
have to plan ahead and make shopping lists, to minimize the number of times they have to go shopping. Furthermore if they do not have time to shop all items on a shopping list, it should be possible to separate the most needed items from items, which can wait a set number of days, to be bought. Examples of people who can benefit from saving time because of tight schedules can be:
\begin{itemize}
\item Students
\item Parents
\item Families
\end{itemize}

\emph{People who want to be social while eating}
When people want to get together and eat for different occasions, it could be beneficial if they plan the meal based on the preferences of the people involved. The reason why people would like to be social while eating could be just wanting to talk to others, but also to save money by preparing bigger meals, and people trying to have less food waste by cooking together. Examples of people who would like to be social while eating can be:
\begin{itemize}
\item Social eaters, people who only get together to eat a meal together
\item Students
\item Parents
\item People with a tight or small budget
\item Students
\item Parents
\end{itemize}

\emph{Comparison}
Looking at people, in different ways, different requirements can be set. Some of these requirements are that, different cultures and ways of understanding certain icons, must be taken into consideration, furthermore it seems to be beneficial to all, to receive help in planning meals, this can be done by shopping lists, and automated processes, to help generate this shopping list. Lastly, it would be beneficial to have help handling different cultural requirements, such as certain food that should be ignored. \fxnote{I am not happy about this comparison, should be revisited, CHS}