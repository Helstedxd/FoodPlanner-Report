\chapter{Future Improvements}
In this chapter the improvements which could be made in the future will be listed and described. A reflection on how the program could be used in other contexts will also be made in order to explore new areas.

\section{Improvements}
The improvements mentioned in \cref{DelimitationSection} should all be included in a full version of the system. These changes should be accompanied by the changes presented in the \nameref{UsabilityTestSection}.

More changes have been thought of while developing the system and writing the documentation. These changes or improvements will be described in this section.

An improvement to the system would be to include a budget page where the user could manage his or her spendings for the month. It should be possible to set a budget for any month, and the program will then generate a meal plan that will fit to the budget. The shopping list should display the total cost of the items present on the list. This change would be beneficial as some users have applications on their smartphones or computers that help them with their budget. By having a part of the user budget in the program, it becomes more of a complete package for the user.

Something else that was thought of would be the possibility to pick or exclude stores to shop in. When the user is browsing their shopping list, they should be able to get recommendation on where to buy the different products. Some of the shops might lie on the users route back from home or their children's school. This feature could be expanded to allow the user to set a maximum distance for how far they would like to travel when shopping. If they set a distance of four kilometres, the program will only look for shops within a four kilometre radius from their home. If the needed ingredients on the shopping list could not be found in the shops, the user should be informed on the problem and recommend other shops that lies outside the radius.

Another considered feature were to include \textit{smart shopping}. The system should look for sales and try to incorporate them into the shopping list so the user would only buy the items when they were on sale or if they were needed soon. Normally the shopping list would only show the items that are within the \textit{days to shop for} time limit, but if an item that is needed later comes on sale, it should also be added to the list. This goes along with the two previous improvements as it becomes easier for the system to stay within the budget, and it allows it to only look for sales in certain shops.

\section{Reflection}
The program's focus could change to bigger groups of people instead of families. A kindergarten could implement the system in order to track what the children should have for lunch. Some of the children could suffer from milk allergy or diabetes and would therefore need special diets. The program should allow the pedagogues to make appropriate servings of lunch to the children, while helping them come up with recipes for alternative dishes. This would make the administrative work easier and minimise food waste in a place with many individuals. In order to make such a system new recipes have to be implemented as the current system is focused on dinner recipes and not lunch recipes. The recipes should also be focused on children and should promote healthy dishes.

The program could also be redesigned to be a website for a catering company. The customers would be people who would order for their parties. They would do so by going to the website and then choose the dishes they would like to order together with the expected guest count. The catering company would then get an order with a list of recipes they would have to cook together with an auto-generated shopping list and price. The price would be based on the shopping list and other factors such as how time consuming the dishes would be to cook. This could double up as being an administrative program that shows the arrangements they would have with their customers. This would help the company save time by automatically making shopping lists and scheduling their calender. The food waste aspect would not be as important, as the amount of food sold, would probably not be affected by using the system. However, it might help the company to not buy unwanted ingredients, as the system would keep track of what is currently stocked. The obstacles that would arise if this system were to be made would be to convert the entire program to be a website instead, even though some of the current features would not be ported. The interface and how to navigate would need to be changed entirely as there would be a separate customer view for the customers, and an administrative view for the staff of the company.