\section{Concrete scenarios evaluation}
This section will present the results that has been deducted from evaluating the different scenario notes and comparing these to one another. The intentions with performing an evaluation of the scenario notes is to uncover Design considerations that has to be addressed. The section is structured with each aspect of the program divided into sections. Each section has some associated windows and each window has design considerations. Each window has some shared design considerations such as layout of information, navigation through information and relevance of information. This is common for all the screens and will not be described as little as possible.

\subsection{Start}
This aspect is implemented in the \textbf{Start screen} and is the first screen a user will encounter. It has to show the user different icons that guides the user into different aspects of the program. The different functionalities for the start screen are as follows.

\begin{itemize}
	\item Start screen.	
		\subitem Enable navigation to Meal calender screen, Recipe browsing screen, Settings screen, 				Shopping list screen, Inventory list screen and Login screen.
\end{itemize}

\subsection{Meal schedule}
This aspect includes the \textbf{Meal schedule screen} and \textbf{Specified day screen}. Meal schedule is a calender like overview of the user's meal plan. It includes planned and unplanned days. The specified day screen is only for on day and shows all the planes for the specified day.

\begin{itemize}
	\item Meal schedule screen.
		\subitem What information needs to be displayed? 
		\subitem Indication of news/ changes.
	\item Specified day screen.
		\subitem Add meal button.
		\subitem View meals.
		\subitem Change meals.
\end{itemize}

\subsection{Recipe}
This aspect includes the \textbf{Browsing recipes screen} and \textbf{Specific recipe screen}. The first screen is used by the user to browse different categories which gives the user an easy overview. The other screen is used to display specific recipes and information such as ingredients, description and cooking guide.

\begin{itemize}
	\item Browsing recipes screen.
		\subitem Different categories. 
		\subitem Indication of recipes that the user has ingredients too.
		\subitem way of browsing.
	\item Specific recipe screen.
		\subitem Relevant information about the recipe.
		\subitem Add to meal plan method e.g. a button.
\end{itemize}

\subsection{Login}
This is a single screen that is used the first the a user starts the program and whenever a user wants to relog.

\begin{itemize}
	\item Loading screen.
		\subitem What information should this screen show/ ask for?
\end{itemize}

\subsection{Setting}
This aspect of the program deals with displaying and enabling the user to modify settings in the program. This aspect has the \textbf{Setting screen}, \textbf{Specified setting category} and \textbf{Change a specific setting}. It is debatable if all of these screens are necessary. This issue will be evaluated through the design process.

\begin{itemize}
	\item Setting screen. 
		\subitem What settings is there?
		\subitem can the settings be categorized?
	\item Specified setting category.
		\subitem What information needs to be shown for each setting?
		\subitem How can the user change these settings?
	\item Change specific setting.
		\subitem How does the user save the changes?
\end{itemize}  

\subsection{Shopping list}
This aspect of the program deals with the display and administration of the shopping list via the \textbf{Shopping list screen} and \textbf{Shopping list new item screen}. These screens gives the user an overview of the shopping list and a way to add new items to the list.

\begin{itemize}
	\item Shopping lists screen.
		\subitem A overview of all ingredients with relevant information.
		\subitem A way to add new items to the list.
		\subitem Ways for the user to modify specific information e.g. item quantities.
	\item Shopping list new item screen.
		\subitem Does the user need to fulfill all information about a new item?
		\subitem How does the user complete the add item?
		\subitem hos does the user undo information? 
\end{itemize}

\subsection{Inventory}
This aspect of the program is used to administrate a users inventory by the user. This has two screens
\textbf{Inventory list screen} and \textbf{Inventory list new item screen} and is similar to the shopping list aspect. 
    
\begin{itemize}
	\item Inventory list screen.
		\subitem Overview of each item in the inventory with information about quantity, expiration and possible more.
		\subitem Way to add new item to the inventory
	\item Inventory list new item screen.
		\subitem The same considerations as the shopping list new item screen.
\end{itemize}


