\section{Interviews}
Interviews have been conducted, on 5 persons in different households, and the answers can be seen in \cref{Interview}. In this section similarities and differences are looked upon.

\subsection{General}
this subsection contain some of the general questions.

\textbf{Sex}

Two participants were male, and 3 were female, which makes both sexes represented.

\textbf{Age}

The two males are the youngest, at 25 and 28 years, the females are 53, 55 and 75. This gives a broadly spread range, representing a wide target group.

\textbf{Household}

Most households are similar in size, with three households of 2, one household of 3, the last household with 11 persons. This gives a generally good idea of problems that can be found in smaller households.

\subsection{Food waste}
The participants see food waste has the same opinion as the FDB participants to food being thrown out, which is that edible food being thrown out is food waste. Another aspect that was answered by the interviewees is the question of using leftovers, which is discussed underneath. Some participants saw food fed to animals, as not being food waste, and lastly a reason is given for the bundles, in which food is sold, are to big, so the household do not have the time to use it all before it goes bad. The participants were asked if they found if food waste was a problem in their household, and only one saw it as an actual problem, it should be noted that two participants fed food, that was about to go bad,  to animals, such as hens or dogs. furthermore if food was thrown out it would mostly be vegetables, and sometimes bread. The participants were asked what they focused on, when they had to decide if food should be thrown out, and all answered that they did not take the best before date into account, but instead looked at the freshness of the food.

\textbf{Leftovers}

All participants say they use left overs, even though the given reasons are different.
\subsection{Importance}
The participants were asked what was the most important about their diet. And the order of their answers varied a lot, but some of the aspects they all had in focus, were quality of the food, and that the prize is good. Other aspects were organic, exiting, healthy and varied meals.
\subsection{Planning}
The participants have in common that they do not plan more then a few days ahead. But they plan their shopping in different ways, one shops for dinner almost every day, another make plans from meat that can be found in the freezer, another is willing to drive far for proper meat, the next wants to shop a few days before the planed meal, and the last person just plan and shop when he arrives at the store.

\textbf{Shopping}

It was hard for some the participants to say, for how long they shop, but two of them shop in 5-10 minutes. Another participant elaborated that she often goes to multiple shops, and therefor the shopping often takes about an hour, in total. Furthermore the participants could not all elaborate when, in the day, they shop, but some said that it would mostly be after work. Next, the participants shop between 3 to 5 times in a week. Only one of then prefer to plan the shopping in advance, by looking through magazines, to find the best offers. One elaborates, that if he knows what he wants before he goes shopping, it will take a lot less time.

\textbf{Shopping situation}

The participants are affected a little differently by impulsive shopping, but in general, do they like to buy things that are on sale. And some likes to buy the food that is on sale for being close to the best before date.