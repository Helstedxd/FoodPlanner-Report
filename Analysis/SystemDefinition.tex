\chapter{System definition}
I this chapter the system will be defined through a system definition.

\section{Definition}
To formulate a system definition we used the FACTOR \cite{BATOFF} criteria to establish important parts of the system.
\fxnote{HUSK at lave kilden BATOFF til systemudvikling.}

\begin{description}
	\item[Functionality] Help users organize a meal plan based on their preferences and inventory. Create/edit user, inventory and plan.
	\item[Application domain] Database responsible employees \fxnote{what is this (explain in definition below)}, Users planning meals.
	\item[Conditions] The system will be used by users with varying technical abilities and cooking experience.
	\item[Technology] Tablet, smartphone
	\item[Objects] Family member (user), inventory (groceries), recipe
	\item[Responsibility] An administrative/planning tool
\end{description}

\subsection{Definition}
A system which primarily lets you plan your meals in advance based on what is in your inventory to assist the user and reduce food waste. Secondarily the program will allow administration of a meal plan and inventory and incorporate functions to alter this plan and inventory by adding or removing recipes and groceries. The system will be designed for tablets and smartphones because these devices offers mobility and thereby available both in and out of the user’s home.
