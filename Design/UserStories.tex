\section{User Stories}

In this section, four user stories are going to be generated, from which conceptional scenarios can be made. The user stories are made with the interviews in mind, so that the user stories derives from reality.

\subsection{Person A}
The person does not plan ahead for long periods, meaning that only 1 day is planned at a time. The planning does not take place at any specific time during the day. The shopping is typically done for the current day and it is not considered what the person has at home before buying groceries. The task of shopping is mostly given to whoever has the time or is out already, since person A is living together with others.

\subsection{Person B}
The person plans a few days at a time by cooking larger quantities, so leftover will be available for later. On the day that the cooking is done, meat is taken from the freezer in the morning to be prepared in the evening and if possible eaten the next days. The shopping is planned when the meat is taken from the freezer so if something unexpected happens, for example there is not any meat, the person will have to shop after work.  
 
\subsection{Person C}
The person plans 2-3 days ahead. When the person shops he/she will buy what the person and family wants to eat, or if there are discounts at the supermarket. Discount catalogues are also looked through when they arrive in the mail, so the person knows what to buy when going grocery shopping.

\subsection{Person D}
The person buys large quantities of meat And based the meat the rest of the ingredients are bought, so that the recipe can be made.
