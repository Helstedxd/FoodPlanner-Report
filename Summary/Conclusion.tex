\chapter{Conclusion}
It is a problem that a consumer wants fresh groceries, and at the same time purchases too many groceries, because they do not want to do much shopping. This problem is attempted solved with a software solution, by generating a shopping list to inform the user about which groceries is needed to be bought according to scheduled meals and current inventory.

It can be complicated to plan a meal schedule for many people, but through the software solution the meals can be scheduled for more people, and the shopping list will be updated accordingly. This is also a trait for people living alone, since they on average have a greater food waste level than people living together. Therefore the ability to plan meals ahead of time, could both lower the food waste level as well as the cost of the meals.

Being on a specific diet can be a complication when shopping, because it can be difficult to find specific ingredients needed for the diet specific recipes. By having a shopping list, which includes the ingredients of the recipes, the user will much faster be able to find the ingredients. The system allows the user to choose a specific diet, which they want to follow. The solution will add ingredients, that should be avoided to a "blacklist", making sure that the recipes including these ingredients will be excluded, as well as rating ingredients to prioritise the search results.

Not knowing which groceries you already have at home, will often increase the food waste level. The software solution allows the user to keep track of their inventory, either by adding ingredients directly to the inventory list, or by adding them from the shopping list after the they have been purchased.

The interaction design of the program was created using the Scenario-based Design method. The method was first used to combine design requirements and a scenario corpus into a conceptual model. Then concrete scenarios were used to create paper prototypes. The conceptual model and paper prototypes were then assembled into a design language, which was followed throughout the design of the program. The design requirements for the program functionality was taken into consideration during development of the solution, however not all of the requirements was fulfilled.
