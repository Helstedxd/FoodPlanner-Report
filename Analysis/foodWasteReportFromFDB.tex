\section{FDB Report}
The following section examines a report published in 2011 by FDB called "Forbrugerne: Vi smider ike mad ud"\cite{madSpild_FDB}.
\subsection{Method used in the report}
The report is an anthropological study of how food waste occurs and how it is experienced by the consumers. The study is based on qualitative data collected from six different households. These households are all different from one another e.g. they live in different geographical locations. Data is gathered from the participants by using observations e.g. watching participants shop or prepare food, semi-structured interviews and probing kits. The data is then used to examine the activity patterns and motivation of the participants.

\subsection{FDB report: what is food waste and how does it occur.}
The report discusses why food waste occurs and how the participants perceive it. There are some subjects that are interesting when learning about food waste. 
\textbf{What is food waste} 
The first subjects discussed what food waste is. When the participants was asked how much food they throw out, they typically say as little as possible. But there is a clear distinction between waste and non-waste being thrown out. It is acceptable to throw waste out, but not to do the same with non-waste. The waste and non-waste categories varies depending on cooked food products and uncooked food products. Cooked food is considered waste when larger parts, or leftovers of a meal that has not been served on a plate is thrown out. So when the food has been served on a plate it is acceptable to throw it out. Uncooked food is considered waste when the packaging is unopened. It is not considered waste when the uncooked fat or other food parts that are considered non-edible is thrown out.
\textit{Cause of food waste} 
The reasons as to why food is thrown out varies depending on cooked and uncooked products. Uncooked products is thrown out because they are bought in large quantities and the participants could not eat it all. Products that are considered non-edible such as cartilage or bad vegetables are also thrown out. Cooked food products are also likely to be thrown out. Cooked food products are thrown out because the participants prepared more than they could eat in one meal. If the participants estimated that there were enough leftovers to save them for later they would. But often the leftovers was stored in the fridge for 4 days only to be thrown out.
\textit{Food waste barriers}
The participants might have the will to reduce food waste but there are some barriers that can impact the participants will in a negative way. The status of the meal ingredients can be a barrier. Dinner is often made from the kind of meat that is used. This means that all other ingredients on a plate only is supplementary. It is easier to throw away the supplements than the meat because of its status. Therefore supplements represents a larger part of the food waste in the homes. Making a large delicious and fresh meal is also a way to show appreciation towards guests or household members. This can also act as a barrier because it requires the food to be fresh and in large amounts. Another barrier is quantity discounts. It contradicts with the participants reasoning when they have to decide between buying what is the right economical, or what is right according to food waste. In the situation people often think of the economical aspects, and not on what they will do with the extra food. The participants also expressed that a lack of overview of what products they had at home when they were out shopping. Sometimes they bought something that they already had at home just because they was not sure if the had it at home. This resulting in an overstocking of a product that in the en most likely would get thrown out. Another barrier was planning versus impulse and desire. Participants that shopped regularly, said that they sometimes changed their minds on what they wanted for dinner or that they wanted variation in their dinners. This spontaneity and need for variation could result in products being bought for one night and the leftovers being stored and in the end being thrown out. Participants that shopped frequently said that they wanted variation and therefore did not plan dinners for a whole week. This could result in a lack of overview and in the end more food waste. The expiration date can also affect the participants willingness to buy a product. The tolerance of when a product is not edible varies between participants, and when the product can give a smell or a visual sign of decay. The expiration date had less importance. The last barrier that is discussed is what is acceptable as food waste. Some of the barriers that has been mentioned is accepted in some degree by the participants. One of the participants talks about the acceptance of food waste when it comes from a child's plate. The participants says that it is hard assess how much the child will eat and that it leads to food waste but it is a waste the participant will accept.