\chapter{Search}

\chapter{PublicQuerys} \label{chp:PublicQuerys}
The class PublicQuerys is a collection of query that can be used to get information from the database. It is also the brain behind the search and sorting.
The fields will be described as they are used several places in the application, so they will be described here.

\section{Fields}
\subsection{inventoryIQueryable \& inventoryList}
The fields inventoryIQueryable \& inventoryList is used to get all the ingredients in the users inventory. This is used in the among other in the search.

What it does is it gets the users inventory and since the inventory can contain multiple of the same item, where the expiration date is different from each item.

Since the program does not remove an ingredient if it gets too old the application can ignore expiration date and group each ingredient into one and combine the sum. The only difference between inventoryIQueryable \& inventoryList is that inventoryList is made into an list so it can be used inside another query without having another datareader open.

\subsection{ingredientsFromLastMeals}
This field gets all the users previous meals and get the ingredients from them. Then group the ingredients into the class LastMeal where the ingredientCount is the number of times the ingredient has been used previously.

\subsection{blackList} \label{ssc:blacklist}

This field retrieves all the blacklisted ingredients from the specific user, and then finds all the recipes where at least one of the blacklisted ingredients is used.

Then it selects the ids of the found recipes to the field. Duplicates is ignored because the outcome dies not change if a recipe is ignored multiple times.

\subsection{grayList}\label{ssc:graylist}
This field retrieves all the graylisted ingredients. The graylist is essentially a rating system for ingredients where the higher the value the ingredient has, the higher priority a recipe with the ingredient has.

When the grayList is used only the first of the specific grayList ingredient is used from the list, the list needs to be grouped by ingredient. But the value of the specific grayList ingredient cannot exceed the max value of 100, so if a duplicate ingredient occurs, which can happen if a diet is set, the grouping needs to handle it. 

This is done by dividing the sum of the ingredient value with the number of time the ingredient is found. 

\section{Methods}
\subsection{search}
This method takes a list of ints as parameter input, the ints are ids that is associated with the recipes, then the method finds all the ingredients who is associated with the found recipes. Then it group it by the recipe id to an IQueryable so the recipes has all the ingredients connected.

\begin{lstlisting}[caption=ResultGrouping, label={lst:ResultGrouping}, language=CSharp]
group new Result()
{
   recipe = ri.Recipe,
   ingredient = ri.Ingredient,
   quantity = ri.Quantity
} by ri.RecipeID;
\end{lstlisting}

\subsection{addValuesToSearch}
This method is the method that adds value to the search results and sorts the list.

The parameter input of the method is a IQueryable, and a list of strings which is initialized to null.

\begin{lstlisting}[caption=addValuesToSearch, label={lst:addValuesToSearch}, language=CSharp]
        public ObservableCollection<SearchResults> addValuesToSearch(IQueryable<IGrouping<int, Result>> userInput, List<string> searchKeywords = null)
\end{lstlisting}

The first foreach that is found is traversing through all the recipes in the list, and for each recipe a \textbf{SearchResults} is initialized.

The list of string may be a part of the parameter input is tested if it's null, if it is the application assumes that the search is not don’t via keyword search. If the list is not null the application increment the \textbf{SearchResults} kewordmatch value.

Then the application traverse through each of the ingredients, where it first adds the ingredient to the \textbf{SearchResults}. There after the application check if the specific ingredient is in users inventory, if the ingredient is, the application checks if the user has all of the ingredient, if the user has the property fullMatch is incrementet in the \textbf{SearchResults} object.

If the ingredient is found but only partial the application adds the percentage match to the partialMatch property in the \textbf{SearchResults} object.

As with the keywords with the recipes, if the keywords is set the application also tests the ingredient for a keyword match.

After that the application checks if the specific ingredient has been used in a previous meal or meals.
If it has the application adds the number of time the ingredient has been used to the prevIngredients which is a property in the \textbf{SearchResults} object.

Then the application check if the ingredient is found in the graylist, if it is it use the rating, if not it sets the rating to 50 which is the default value.

After all this is done it adds the \textbf{SearchResults} object to a ObservableCollection which is returned.