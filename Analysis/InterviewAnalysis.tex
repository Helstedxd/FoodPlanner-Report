\section{Interviews}\label{InterviewAnalysis}
Five persons from different households participated in interviews, and the answers can be seen in \cref{Interview}. In this section similarities and differences are looked upon.

\subsection{General}
This subsection contain some of the general questions.

\textbf{Sex:}
Two participants were male, and three were female, which makes both sexes represented.

\textbf{Age:}
The two males are the youngest, at 25 and 28 years, the females are 53, 55 and 75. This gives a broadly spread range, representing a wide target group.

\textbf{Household:}
Most households are similar in size, with three households of two, one household of three, the last household with 11 persons. This gives a generally good idea of problems that can be found in smaller households.

\subsection{Food Waste}
The participants' opinion on food being thrown out, is the same as for the participants in the FDB report, which is that edible food being thrown out is food waste. Another aspect was that
all participants said that they used leftovers. Some participants saw food fed to animals, as not being food waste, and lastly a reason is given for the bundles, in which some food is sold, are to big, so the household do not have the time to use it all before it goes bad. The participants were asked if they found food waste to be a problem in their household, and only one saw it as an actual problem, it should be noted that two participants fed animals with food that was about to expire. Furthermore if food was thrown out it would mostly be vegetables, and sometimes bread. The participants were asked what they focused on, when they had to decide if food should be thrown out, and all answered that they did not take the best before date into account, but instead looked at the freshness of the food.

\subsection{Importance}
The participants were asked what was the most important about their diet. The priority varied a lot among the participants, but some of the aspects they all had in focus, were quality of the food, and price. Other aspects were: organic, exciting, healthy and varied meals.

\subsection{Planning}
The participants have in common that they do not plan more than a few days ahead. But they plan their shopping in different ways:
\begin{itemize}
\item Shops for dinner almost every day.
\item Make plans from meat that can be found in the freezer.
\item Drive far for proper meat.
\item Shop a few days before the planed meal.
\item Plan and shop when arriving at the store.
\end{itemize}

\subsubsection{Shopping}
It was difficult for some the participants to say, for how long they shop, but two of them shop in 5-10 minutes. Another participant elaborated that she often goes to multiple shops, and therefor the shopping often takes about an hour, in total. Furthermore the participants could not all elaborate when, in the day, they shop, but some said that it would mostly be after work. The participants shop between 3-5 times in a week. Only one of them prefer to plan the shopping in advance, by looking through magazines, to find the best offers. One elaborates, that if he knows what he wants before he goes shopping, it will take a lot less time.

\subsubsection{Shopping situation}
The participants are affected a little differently by impulsive shopping, but in general they like to buy things that are on sale. Some likes to buy the food on sale for being close to expiration.