\chapter{Delimitation}\label{DelimitationSection}

In this section the delimitation of the program is done. When the report was written and the sources were found, a lot of good ideas for the program was discussed. These ideas were then written in the report, so to make the program, it was needed to delimit from some of the ideas, as the program might either become too complex or too time craving to make.

\section{Not in program}

In \cref{HowToAvoidFoodWaste} it is mentioned that incorporating leftovers is a good way to avoid food waste, according to The Danish Ministry of Environment. This is partially done in the program. In the program, when the user prepares a meal, the ingredients used will be taken away from the users inventory. If the user have leftover from the meal, they can add them to the inventory, and if these leftovers are needed to make some of the recipes incorporated in the program, these will then show up higher when meals are being planning.

In \cref{HowToAvoidFoodWaste} there is another initiative to reduce food waste, and that is by making stores available to sell quantity discount, but for the user to pick the items up at different times. This is not incorporated in the program, as this would require the project to have one additional focus point than meal planning.

In \cref{HowToAvoidFoodWaste} another suggestion for reducing food waste is to inform the user of different methods to see if the food has gone bad. This is not done in the program. The program does keep track of the expiration date of the items in the users inventory, but does not inform the user of different ways to see if the food has gone bad, as some of these dates might be guidance and not actual expiration of the food. It was suggested to to colour the food information text in the program when an item expired, this was never implemented in time.

In the section \ref{HowToAvoidFoodWaste} it is also mentioned that if it where possible to share quantity discounts with other people, food waste would be reduced. This is not incorporated into the program because the focus of the report is on meal planning, and therefore it would require a complete shift of focus to make a program that was able to do this.

By keeping track of what is being thrown out, it can be seen if the same type of ingredients are being thrown out often, and the user might need to buy less of this ingredient it is suggested in \cref{HowToAvoidFoodWaste}. This is not implemented in the program, as the program only tells the user to shop for the items needed for the recipes, and therefore the shopping list will only consist of needed items. 

It is mentioned in \cref{HowToAvoidFoodWaste} that by being creative with the leftovers, and making meals, from leftovers of which is already parts of a meal, food waste can be reduced. This is partially incorporated in the program, as the user can put in the leftovers in the inventory. If a recipe uses these leftovers, recipes including these will be listed higher, when the user searches for a recipe to schedule. Though there is not a specific initiative to use leftovers from meals creatively.

In \cref{SituationLabel} it is mentioned how a conflict can arise when multiple users try to change the same meal plan. The function of multiple users using the same meal plan, has not been incorporated in the program though. The reason for this is that it would require to expand the program, and that time for this was not available.

Section \ref{IdeasLabel} has some ideas for the program. Not all of these were used. The section suggest the availability to create custom recipes through the program, or buying recipes for the program as a from of program expansion. This feature has not been created for the food planner program, as the recipes in the program are taken from a website\Cref{BigOven}, and are freely available on for the public to use.

In \cref{IdeasLabel} it is also mentioned that a user should be able to register for an account to synchronize between different devices. This is not incorporated, as the time for the creating this feature was not available.

In \cref{IdeasLabel} it is mentioned that a "surprise function" is an idea already implemented by other meal planning services. This feature creates a meal plan for the user, without the user needing to plan it themself. This function is not incorporated in the program. However, there are recommended recipes in the search page based on the users information.

Section \ref{IdeasLabel} also suggests the user being able to exclude recipes that takes a specific time or more to cook, so the user can choose to cook recipes that require a specific time or less. This is not done in the program, because the website where of the recipes have been taken does not include the time it takes to cook the recipe.

Section \ref{IdeasLabel} mentions that by providing the user with inspiration for leftovers, food waste can be lowered. This is not incorporated in the program, though the user can add leftovers to the inventory list, if the leftovers ingredient type is present in the database.

To have featured meal plans for the user to gather inspiration or for the user to choose from is mentioned in \cref{IdeasLabel}. This is not made in the program, as the program only contains one meal plan. The user can get the inspiration from the recipes screen.

Customer plans is an idea mentioned in \cref{IdeasLabel}, it means that the user should be able to browse through other users food plans for inspiration. This is not incorporated in the program as there is no online community made available in the application.

"My binder" is an idea mentioned in \cref{IdeasLabel} which suggests that the user should be able to bind recipes they like, so they can find them later easily. This is not in the program, as the project is used to lower food waste by giving the user the ability to find the recipes with the most ingredients already owned.

\section{In program}

In \cref{HowToAvoidFoodWaste} it is also suggested that by following recipes and using the exact amount of the ingredients will reduce the food waste. This has been incorporated into the program as a user has the ability to change how many people the recipe is for, and then the amount of the ingredients needed, also changes.

In \cref{HowToAvoidFoodWaste} it is also suggested using the items that expires soonest. This is part of the program, as the searching of adding a new recipe to the meal plan takes into consideration, and sorts the recipes so that the ones with ingredients expiring sooner gets higher up.

In \cref{FoodPlanStudyReflection} it is mentioned that people wanted realistic food, not food from cookbooks, because of the time and resources needed for making those. This has been taken into consideration in the program. The recipes from the program are stripped from a website, these recipes vary in time and easiness to make. The searching when planning a meal shows how many of the ingredients you already own, so you do not need to go shopping for a lot of ingredients maybe to make a recipe, as you can see if you already have what is needed to make the meal.

In \cref{FoodPlanStudyReflection}, four criterias was found when people were making a meal plan: Food you already have, food you like, price, and stock of stores nearby. This is partially incorporated in the program. As described previously, food the user already own is considered when planning a meal. In the settings page you can rate ingredients, so you can choose ingredients you like more, and they will be prioritized higher, when searching for meals to schedule on your meal plan. The price of the food, could not be obtained, as no available database where found which included prices for ingredients. Stock of stores nearby was not possible either, as supermarkets do not let their stock be available to the public.

Section \ref{IdeasLabel} has a list of ideas for the food planning program, that are inspired by other programs. One of the ideas is the inventory that keeps track of the ingredients a user need. This is included in the program as Stock Ingredients. 

In \cref{IdeasLabel} it is also mentioned that the availability to create meal plans is an idea for the program. This have been made in the program, as the user are able to create a meal plan, that meals can be scheduled in. This is combined with the "Planned Meals" idea mentioned in \cref{IdeasLabel}.

\Cref{IdeasLabel} also mentions how the user should be able to add groceries to a shopping list. This is a feature in the program, as the items needed for the recipes are added automatically to the grocery list.

\Cref{IdeasLabel} presents the idea of the user being able to plan both breakfast, lunch, and dinner on the same day. This is incorporated in the program, as the user is able to plan as many meals as needed.

In \cref{IdeasLabel} it is mentioned that the user should be able to choose some dietary preferences. This has been made in the program, as some test diets have been set up, so the user can choose the specific diet he/she is following. The rating system and blacklisting system also allows the user to choose which ingredients he or she really like to use or do not want to incorporate in their meal plan, which allows for the user to make a diet more specific.