\section{Concrete scenarios}
\textbf{*intro*}

\subsection{MealPlan/01} \label{MealPlan01}

In this scenario a user wants to plan three meals for three days. The person does this by choosing a day and then a recipe three times. The person does this at home as she usually does it there.

Title: Three day planning via days

Overview: 

	People = Peter, average smartphone user. Works at an office
	
	Activities = Planning meals
	
	Context = In his home. More particularly in his living room.
	
	Technology = His smartphone.
	
\textbf{Rationale}

In this scenario a person wants to plan three days while sitting at home. He will use his smartphone's touch screen when planning. The scenario focuses on the steps he takes to plan the three days.
	
	A1. It is 10:30 and peter would usually be at work this time a day but today is Saturday. He just ate breakfast and remembered that he had to plan what he was gonna eat the three next days. He knows that he feels like eating steaks one of the days but the two other days is not decided yet.
	
	A2. He leaves the comfy sofa to get his smartphone from the charging station and walks back to the sofa. He starts searching for the program on his smartphone and starts it when he finds it.
	
	A3. On the start screen of the app he briefly views the different options[1]. He chooses to touch the view meal plan icon[2] and gets directed to an overview of his meal plan[3]. 
	
	A4. on the next screen he gets a display of the next few days that he can plan. He starts by touching the next day and gets a detailed view of what meal is planned[4]. Currently there are no meals planned so he finds the "add meal" buttons[5] and gets directed to a new window[6]. 
	
	A5. This window shows him an overview of different categorized recipes[7]. He starts browsing through the recipes until he finds a steak recipe[8]. 
	
	A6. He is interested in this recipe and touches the picture on the screen[9]. He then gets an overview of the recipe including a description, ingredients and how to guide[10]. He then decides to add this recipe to the first day in his meal plan by touching the add to meal plan button[11].
	
	A7. He now gets an updated view of his meal plan with the recently added recipe on the first day[12]. He is satisfied with his choice and now moves on to the next day. 
	
	A8. He now follows the same procedure for the rest of the days that he wants to plan. When he has planned all the days he exists the program and walks into his study room to do some work.
	
\textbf{Notes to scenario MealPlan/01}

1 What options is there and how many? it is important to consider to avoid screen clutter.

2 How are these icons designed? icon/symbol, colour theme etc.

3 It should be considered what information to show e.g. number of days, day information and the visual display etc.

4 what information is important for the user to see here? meal titel, ingredients, quantity etc.

5 how dos this button look? is it even a button or a swipe bar.

6 should it jump between different windows or something else? (this can be critized elsewhere)

7 what categories and how are the criterias they have?

8 how does the browsing work (scrolling, buttons)?

9 should the picture be pressed or dragged or something else?

10 what information should be displayed and how is it arranged on the screen?

11 same consideration as note 9

12 is there a refresh button or does it update on the fly?

\subsection{MealPlan/02} \label{MealPlan02}

Title: One day planning via recipes

Overview:

	People = Maria, average smartphone user, Studies economics at a university. 
	
	Activities = Planning a meal.

	Context = Outside of the home. More particularly on the way home from the university.

	Technology = Smartphone.

\textbf{Rationale}

In this scenario a person wants to plan a meal by choosing a recipe that is interesting. The person does this by choosing a recipe and then adding it t a day. The person does this from a location outside of the home. 

	B1. It has been a long and stress full day for Maria at the university. She wanted to plan her meal at lunch but did not find any time to do it. She is now on her way home in the bus and finally feels that she can figure out what she wants to eat tonight.

	B2. She opens her purse and pulls up her smartphone. She quickly checks her e-mail and afterwards finds the meal planning program and starts it[1]. 

	B3. On the start screen she is presented with different icons[2]. She touches the recipe icon and a new window appears.

	B4. The screen now shows her many different recipes[3] and options to browse through[4]. She starts to browse[5] them looking for what she wants to eat. 

	B5. She finds a recipe that she is interested in and touches the icon for that recipe[6]. A more detailed overview of the recipe appears[7] and she quickly scrolls through the overview[8].

	B6. After reading about the recipe she decides not to have it. She now goes back to the recipe browsing screen[9] and continues her search.

	B7. She now finds a new recipe that she wants to view. She touches the recipe picture and gets an overview of that recipe. 

	B8. She decides that this is the recipe that she wants. She now touches the add to meal plan button[10] and gets some options to associate the recipe with a specific day[11].

	B9. She is now satisfied with her choice and closes the program and start wondering what home reading she has to do for tomorrow.
	
\textbf{Notes to scenario MealPlan/02}

1. Is there any login?

2. Same issue as MealPlan/01 note 1.

3. How is these shown (pictures, text, visual display)?

4. In which ways can the user navigate?

5. How does the browsing work?

6. how do the user select a recipe? does it show if she have had this before or it is on her meal 
plan?

7. What information is shown in the detailed overview?

8. note 4 and 5 applies here

9. how does the "back" function work?

10. is this a button or something else?

11. How does the user adds this meal to a specific day (new screen, drop down menu)?

\subsection{MealPlan/03} \label{MealPlan03}

Title: Shopping using the program's shopping list.

Overview:

	People = Susan, skilled smartphone user. works as a editor at a local TV station. 
	
	Activities = Viewing the shopping list and update the shopping list.

	Context = Outside of the home. More particularly in a grocery store.

	Technology = Smartphone
	
\textbf{Rationale}

In this scenario Susan has just gotten off work and is on her way home. Before she drives home she needs to go to the local grocery shop. She has made a list before hand by using the meal planning program.

	C1. Susan has gotten of early from work today so she is in no rush to get home. Before she gets into her car she remembers that she needs to go grocery shopping because her smarthpone just vibrated in her pocket. When she goes into the grocery shop she picks up a basket and takes her smartphone up from her pocket and finds the meal planning program.
	
	C2. On the start screen she quickly finds the shopping list icon[1]. She touches this and a list with all the groceries and the quantities she needs is shown[2].
	
	C3. As she goes around getting each item on the list she checks off each item on the list[3].
	
	C4. When she gets to the meat item on her list she decides to buy some extra meat because she knows it will be eaten. She does this by touching the quantity indication on the meal item[4] and changing it[5]. 	
	
	C4. As she goes by the dairy products she remembers that she also wants some more milk. She picks up two liters and then looks to add it to the shopping list.
	
	C5. She touches the add item to list button and a new screen appears[6]. She selects the product she wants to add and the quantity of it[7]. When she has done this she selects complete[8] and goes back to the shopping list which is now updated.
	
	C6. After getting all the items on the list she touches the shopping list complete button[9] and goes up to the register and pays. She then drives home knowing that she has gotten all that she needed for her meal plan.    
	
\textbf{Notes to scenario MealPlan/03}

1. Is this icon on the main screen?

2. How is each item shown(by text pictures) and what information is shown to each item.

3. Does she check of each item or is it unnecessary?

4. How is the quantity changed on each shopping list item?  

5. How is the changing option performed by the user?

6. Is there also a remove item? does a new screen appear and what does the screen show/contain? moving between screens?

7. is these options selected from a drop down menu or by using text?

8. How is this navigation done?

9. How does she indicate that the shopping list is completed?

\subsection{MealPlan/04} \label{MealPlan04}

Title: Changing a specific meal

Overview:
	
	People = John, average smartphone user, Works as a teacher.
	
	Activity = Viewing the meal plan and editing the meal plan.
	
	Context = In his home. 
	
	Technology = Smartphone.
	
\textbf{Rationale}

In this scenario John wants to view his meal plan and change a a specific meal. He is doing this in his own home.

	D1. It is weekend for John and his parents are coming over for dinner tonight. He was told that his parents would come yesterday and he now wants to change his meal plan because he wants to cook something special.
	
	D2. He finds his smartphone and starts the meal planning program. On the start screen he selects the overview of meal plan icon[1].
	
	D3. He is now shown all of his planned recipes and quickly finds the meal that he has planned tonight[2].
	
	D4. He touches the meal and is shown some details about it[3]. He finds the replace button and touches it[4].
	
	D5. He is now shown the recipe window where he can select a new recipe for that day[5]. He browses through the recipes and selects the one he wants. 
	
	D6. He is now shown a more detailed description of the recipe[6]. He likes that recipe and selects the add button[7].
	
	D7 He is now directed back to the overview of the day and sees that the new recipe is added to the day he wanted.
	
	D8. He exits the program and puts his smartphone away. He now decides to start cleaning the house a bit to prepare for the evening dinner.
	
\textbf{Notes to scenario MealPlan/04}

1. Is this the way to access the change meal plan?

2. How is the meals shown and what information is shown? 

3. How does he navigate to the specified information about a day? and what information is required to be shown?

4. How should he be able to change a meal?

5. Is it the ordinary recipe window or is it altered a bit? (what if doe not want to go shopping) 

6. Same description as anywhere else in the program?

7. How does he add this recipe? the day is chosen so it is not like it has been done elsewhere in the program.

\subsection{MealPlan/05} \label{MealPlan05}

Title: Adding items to the inventory.

Overview:
	People = Ann, Average Smartphone user. Working as a nurse.
	Activities = View inventory and add item to inventory.
	Context = In her home.
	Technology = Smartphone

\textbf{Rationale}

Ann has a day off and is visited by one of her good friends. After the friend leaves she wants to add the fresh apples that she has gotten from the friend to her inventory.

	E1. Ann has had her friend on visit for a couple of hours and now the friend has to leave. Before the friends leaves she gives Ann 1 kg of organic apples from her garden. Ann thanks her and they say their goodbyes.
	E2. Ann walks into her kitchen and puts the apples into on of the kitchen cabinets. She now remembers that she needs to add this item to the inventory list on her smartphone.
	E3. She finds her smartphone and starts the meal planning program. On the start screen she locates the inventory icon and touches it[1].
	E4. A list of all inventory items are displayed on the screen[2]. She starts looking over the screen for the add item button[3].
	E5. She finds the button and touches it[4]. A new screen is now displayed with all the information that filled out[5]. 
	E6. Ann fulfills all the information and touches the done button[6]. She is now brought back to the inventory list screen and can see the new item on the list[7].
	E7. She exists the program and feels sitting down and watching a movie.

\textbf{Notes on scenario MealPlan/05}

1. Is this icon on the start screen and where is it located(size etc.)?

2. What information is shown to each item and how is it shown?

3. How does she navigate on the screen and how is an item added?

4. Is it a button?

5. What information is required and how is it shown on the screen(in a scheme or menus)?

6. Is it just a button?

7. How is it indicated that a new item is on the list both one she adds and anything else new?