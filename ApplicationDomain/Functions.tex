%Funktioner - beskrivelse af systemets funktionalitet se kap 7
%	Komplet funktionsliste
%	Specifikation af funktioner
	
	\chapter{Functions}
	\section{Function definitions}
	The diagrams in \ref{Usage} contain different functions these functions have been given a name and listed in the following tables, where they have been assigned a complexity of Simple, Medium or Complex, and they have been assigned one or more of the following types; Read, Update, Calculate or Signaling.
	\begin{table}[H]
	\centering
	\caption{Shopping list}
		\begin{tabular}{|l|l|l|}\hline
		\textbf{Function}&\textbf{Complexity}&\textbf{Type}\\\hline
	  Update shopping list  &  ???     & Update \\\hline
	  Add item              &  Medium  & Update \\\hline
	  Remove item           &  Medium  & Update \\\hline
	  View shopping list    &  Simple  & Read   \\\hline
	  \end{tabular}
	\linebreak
	\caption{Food plan}
	  \begin{tabular}{|l|l|l|}\hline
		\textbf{Function}&\textbf{Complexity}&\textbf{Type}\\\hline
	  Add meal              &  Complex & Calculate, update \\\hline
	  Remove meal           &  Simple  & Update            \\\hline
	  View food plan        &  Simple  & Read              \\\hline
	  \end{tabular}
	\linebreak
	\caption{Inventory}
	  \begin{tabular}{|l|l|l|}\hline
		\textbf{Function}&\textbf{Complexity}&\textbf{Type}\\\hline
	  View inventory        &  Simple  & Read   \\\hline
	  Add Item              &  Complex & Update \\\hline
	  Edit Item             &  Complex & Update \\\hline
	  Remove Item           &  Simple  & Update \\\hline
\end{tabular}	  	  
	\end{table}
	\fxnote{Christian yo, når du beskriver 'edit' så nævn at det både er 'change ex date' og 'change quantity'}