\subsubsection{Concrete scenarios evaluation} 
The following text will present the results found by evaluating the different scenario notes. The intentions with performing an evaluation of the scenario notes is to uncover design considerations that has to be addressed. The section is structured so that the domains identified in \cref{ScenarioCorpus}, will be used to categorize the different design considerations. If a domain does not have any associated design considerations it will not be used.  

\emph{Planning}

This domain is associated with design considerations from concrete scenario 1, 2, 4 and 6. The considerations are as follows:

\begin{enumerate}
	\item Enabling the user to specify and find the right recipe.	
	\item Indication the number of ingredients the user has for a meal.
	\item How is a meal planned e.g. number of steps.
	\item Making the addition of planned meal ingredients to the shopping list automated by the program.
	\item How does the user add a meal to a day on his meal plan e.g. using a button or dragging.
	\item How does the user search for a specific recipe?
	\item how many criteria or search options does the user have when searching for recipes?
\end{enumerate}


%1 6; How does the program enable Peter to specify and find the right recipe that he wants?
%6; Is there some indication whether or not the user has all the items for a meal and are the items subtracted from inventory?
%1; How does the program make it easy for the user to plan a meal at any time e.g. by minimizing the number of steps to plan a meal
%2; Can the user add/ edit recipes in the program with their one variants of the meal?
%4 6; Could the user pick a recipe and then the ingredients will be added to the shopping list?
%6;  How is does the user add a meal to a specific day in his meal plan via. button, drag?
%6;  How does the user search for specific recipes?
%6; Can the user search for different parameters on one search?


\emph{Shopping}

This domain is associated with design considerations from concrete scenario 1, 2, 3, 4 and 5. The considerations are as follows:

\begin{enumerate}
	\item Can the user how many days to shop for?
	\item How does the user "buy" some ingredients on the shopping list but not all?
	\item What information is shown for each ingredient on the shopping list and how is it shown?
	\item Should the user cross off items on the shopping list or how is it indicated that an ingredient is bought?
	\item Can the user add items to the shopping list?
	\item What information should the user input when adding an ingredient to the shopping list?
	\item Can multiple ingredients be added to the shopping list at once?   
\end{enumerate}
%(5) Can the user specify how many days to shop for?
%How does the user "buy" some ingredients on the shopping list but not all.
%(3) What information is shown to each ingredient on the list and how is it presented?

%(1,2,3,5) Should the user cross off items on the shopping list or how is it indicated that an ingredient is bought?
%(1,2,3) Can the user add items via the shopping list or should they be added directly to the inventory?
%(4,5) Should the user input anything else than name and quantity of the ingredient when adding it to the shopping list?
%(4) Can multiple ingredients be added to the shopping list at the same time?

\emph{Inventory}

This domain is associated with design considerations from concrete scenario 1, 2 and 3. The considerations are as follows:

\begin{enumerate}
	\item What information is shown to each ingedient on the inventory list and how is it presented? 
	\item How is this list kept up to date? can the user add items to the inventory list?
\end{enumerate}
%What information is shown to each ingredient on the list and how is it presented?
%(1,2,3) How does the program make is possible for users to see what items they have at home so they avoid being in doubt?
%(1,2,3) Can the user add items via the shopping list or should they be added directly to the inventory?

%\subsection{Cooking}

%\subsection{General}

%\subsection{Start}
%This aspect is implemented in the \textbf{Start screen} and is the first screen a user will encounter. It has to show the user different icons that guides the user into different aspects of the program. The different functionalities for the start screen are as follows.

%\begin{itemize}
%	\item Start screen.	
%		\subitem Enable navigation to Meal calender screen, Recipe browsing screen, Settings screen, 	%			Shopping list screen, Inventory list screen and Login screen.
%\end{itemize}

%\subsection{Meal schedule}
%This aspect includes the \textbf{Meal schedule screen} and \textbf{Specified day screen}. Meal schedule is a calender like overview of the user's meal plan. It includes planned and unplanned days. The specified day screen is only for one day and shows all the scheduled meals for the specified day.

%\begin{itemize}
%	\item Meal schedule screen.
%		\subitem What information needs to be displayed? 
%		\subitem Indication of news/ changes.
%%	\item Specified day screen.
	%	\subitem Add meal button.
	%	\subitem View meals.
	%	\subitem Change meals.
%\end{itemize}

%\subsection{Recipe}
%This aspect includes the \textbf{Browsing recipes screen} and \textbf{Specific recipe screen}. The first screen is used by the user to browse different categories which gives the user an easy overview of these. The other screen is used to display specific recipes and information such as ingredients, description and cooking guide.

%\begin{itemize}
%	\item Browsing recipes screen.
%		\subitem Different categories. 
%		\subitem Indication of recipes that the user has ingredients for.
%		\subitem Way of browsing.
%		\subitem Search function.
%	\item Specific recipe screen.
%		\subitem Relevant information about the recipe.
%		\subitem Add to meal plan method e.g. a button.
%\end{itemize}

%\subsection{Login}
%This is a single screen that is used the first the a user starts the program and whenever a user wants to relog.

%\begin{itemize}
%	\item Loading screen.
%		\subitem What information should this screen show/ ask for?
%\end{itemize}

%\subsection{Settings}
%This aspect of the program deals with displaying and enabling the user to modify settings in the program. This aspect has the \textbf{Settings screen}, \textbf{Specified setting category} and \textbf{Change a specific setting}. It is debatable if all of these screens are necessary. This issue will be evaluated through the design process.

%\begin{itemize}
%	\item Settings screen. 
%		\subitem What settings is there?
%		\subitem Can the settings be categorized?
%	\item Specified setting category.
%		\subitem What information needs to be shown for each setting?
%		\subitem How can the user change these settings?
%	\item Change specific setting.
%		\subitem How does the user save the changes?
%\end{itemize}  

%\subsection{Shopping list}
%This aspect of the program deals with the display and administration of the shopping list via the \textbf{Shopping list screen} and \textbf{Shopping list new item screen}. These screens gives the user an overview of the shopping list and a way to add new items to the list.

%\begin{itemize}
%	\item Shopping list screen.
%		\subitem An overview of all ingredients with relevant information.
%		\subitem A way to add new items to the list.
%		\subitem Ways for the user to modify specific information e.g. item quantities.
%	\item Shopping list new item screen.
%		\subitem Does the user need to fill in all information about a new item?
%		\subitem How does the user complete the add item?
%		\subitem How does the user undo information? 
%\end{itemize}

%\subsection{Inventory}
%This aspect of the program is used by the user to administrate the inventory. This has two screens
%\textbf{Inventory list screen} and \textbf{Inventory list new item screen} and is similar to the shopping list aspect. 
    
%\begin{itemize}
%	\item Inventory list screen.
%		\subitem Overview of each item in the inventory with information about quantity, expiration date and possibly more.
%		\subitem Way to add new item to the inventory
%		\subitem Search function.
%	\item Inventory list new item screen.
%		\subitem The same considerations as the \textbf{shopping list new item screen}.
%\end{itemize}

%\subsection{Navigation diagram}


%\begin{figure}[H]
%	\includegraphics[width=0.8\textwidth]{Grafik/FoodPlanner/NavigationsDiagram}
%	\caption{This figure shows a navigation diagram for the different screens that have been uncovered in the evaluation process.}
%	\label{NavigationDiagram}
%\end{figure}

%Figure \cref{NavigationDiagram} shows all of the screens that we have found so far through the evaluation of the concrete scenarios. These screens will be sketched in the next section of the report and the content and design considerations of each screen will be specified.  
