\section{Waterfall method}

In the group it was chosen to use the waterfall method for the project. In this section the reason for this choice and a general description of the waterfall method will be given, with focus on the groups use of the method. The waterfall method means that you start in one end of the product and finishes in the other, without going back to a part when you have finished it, where as the iterative approach focuses on going in a circle, and doing the project many times, expanding the product and report each time.

The waterfall method includes 5 steps:

\begin{itemize}
	\item Analysis
	\item Design
	\item Implementation
	\item System test
	\item In operation
\end{itemize}

\textbf{The analysis part} of the waterfall method focus on getting to know your target audience. This is the part of the report that is called problem analysis, which includes the sources of information that was found. In the part, the specific goal of the product was defined, and the part ended out with the requirement specifications. It is also in this section that questionnaires and interviews are conducted.

\textbf{The design part} of the waterfall method is focused on the design of the software. In the project, the design part focuses on the principles learned in the course "Design and Evaluation of User Interfaces". This includes things like a PACT analysis. At the end of the design part, the design specifications are found.

\textbf{The implementation part} of the system, is the part in which the software is created. The design specification is used to make the design, and the needed functionality which was discovered in the analysis as being the requirement specifications is implemented in the product. The outcome of this is the system itself.

\textbf{The system test} is the fourth part of the waterfall method, in this part the system is tested to see if it works, and fulfills the requirements. The outcome of this part of the method is a test report, which is included in the report.\fxnote{Insert reference to where in the report, when this is known.}

\textbf{The system in operation} is the last part. When the system is public, and the outcome of this is a status of the operation. This part though is not implemented in this system, as the product the group makes will not be available for the public to use.

The reason that the waterfall method was chosen for this project, was that a lot of new information would not come throughout the project. Food waste is and will be a problem, and some sort of product is needed to change this. Therefore it seemed natural to gather all the information at first, and work out from that.