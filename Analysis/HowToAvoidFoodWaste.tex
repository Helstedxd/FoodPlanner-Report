\section{How to avoid food waste}

According to The Danish Ministry of the Environment, one of the initiatives you can take against food waste, is to make recipes which incorporate the leftovers, or what is left in the refrigerator. By doing this food that might otherwise go bad, will be used instead of being thrown away. %http://mindremadspild.dk/files/mindremadspild_idekatalog_150611.pdf

Another Initiative that can be taken according to The Danish Ministry of the Environment, is to sell quantity discount, with the option to recieve some of the groceries later. An example would be if 1 cucumber is sold for 10 DKK, while 2 is sold for 15, some people would choose to buy 2 cucumbers because of the quantity discount, even if they just needed 1 at the moment. With a "recieve later" option, you would be able to buy 2 cucumbers, but only recieve 1 at the moment. Then you would be able to come back to the store later to get the other cucumber when you needed it. An initiative like this have already been made by the English supermarket chain Tesco. The initiative is called "Buy One Get One Free Later", and is an initiative for fresh groceries like fruit and vegetables.

To give the consumer better information about when food is going bad is also an initiative that The Danish Ministry of Environment is recommending. By informing consumers of what "Best before", "Last day of sale" and so on means, would get the consumer to use the food more properly. Also by getting the consumer to rely more on smell, taste and looks, rather than just the date of production on the groceries, would lead to lesser food waste.

The Danish Ministry of the Environment, also reccomend getting discounts even thoug you don't buy in quantity. Being able to share an offer with another person would let the supermarkets get volumes bought, and the consumer would be able to only aquire the amount neccessairy.

Only using the gorecries neccessairy is also a way to achieve lesser food waste. Following Recipes will help with this if the recipes is specific. If for example the repicipe is for 4 persons, and 4 persons will be eating, if the recipe is correct food will not be wasted, and even if there are leftovers after the meal, using these as lunch for the next day, or eating the leftovers the day after, will achieve a lesser waste of food. %http://www.brugmerespildmindre.dk/mindre-madspild

Keeping track of products that is being thrown out is another way to reduce food waste. If a pattern occurs you may be overbuying food. An example is if you throw out 500 grams of beef each week, you may be overbuying beef, and might as well buy 500 grams of beef less each week. %http://greatist.com/health/how-to-ways-reduce-food-waste

To eat the food which is closets to the expiration date is a good way to prevent food waste as well. If the refrigurator is being filled up as the week goes by, new products might be put in the front as they are bought last, but by keeping track of the expiration dates by either sorting the refrigurator to have the oldest products in the front, or by writing down items expiration date, would help solve the food waste problem, as the products closest to the expiration date would be used first.

