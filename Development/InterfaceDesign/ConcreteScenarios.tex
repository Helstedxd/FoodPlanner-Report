\section{Concrete scenarios}
\texiti{*intro*}

\paragraph{MealPlan/01}
In this scenario a user wants to plan three meals. The person does this at home as she usually does it there.

Title: Three day planning
Scenario type: Concrete scenario
Overview: 
	People = Peter Nielsen, smartphone-literate. Works at an office
	Activities = Planning meals
	Context = In his home. More particularly in his living room.
	Technology = His smartphone.
	
Rationale

In this scenario a person wants to plan three days while sitting at home. He will use his smartphone's touch screen when planning. The scenario focuses on the steps he takes to plan the three days.
	A1. It is 10:30 and peter would usually be at work this time a day but today is Saturday. He just ate breakfast and remembered that he had to plan what he was gonna eat the three next days. He knows that he feels like eating steaks one of the days but the two other days is not decided yet.
	A2. He leaves the comfy sofa to get his smartphone from the charging station and walks back to the sofa. He starts searching for the program on his smartphone and starts it when he finds it.
	A3. On the start screen of the app he briefly views the different options[1]. He chooses to touch the view meal plan icon[2] and gets directed to an overview of his meal plan[3]. 
	A4. on the next screen he gets a display of the next few days that he can plan[4]. He starts by touching the next day and gets a detailed view of what meal is planned[5]. Currently there are no meals planned so he finds the "add meal" buttons[6] and gets directed to a new window[7]. 
	A5. This window shows him an overview of different categorized recipes[8]. He starts browsing through the recipes until he finds a steak recipe[9]. 
	A5. He is interested in this recipe and touches the picture on the screen[10]. He then gets an overview of the recipe including a description, ingredients and how to guide[11]. He then decides to add this recipe to the first day in his meal plan by touching the add to meal plan button[12].
	A6. He now gets an updated view of his meal plan with the recently added recipe on the first day[13]. He is satisfied with his choice and now moves on to the next day. 
	A7. He now follows the same procedure for the rest of the days that he wants to plan. When he has planned all the days he exists the program and walks into his study room to do some work.
	
\paragraph{Notes to scenario MealPlan/01}
1 What options is there and how many? it is important to consider to avoid screen clutter.
2 How are these icons designed? icon/symbol, colour theme etc.
3 It should be considered what information to show e.g. number of days, 
	  