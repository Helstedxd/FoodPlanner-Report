\chapter{Conclusion}

It is a problem that a consumer wants fresh groceries, and at the same time purchases too many groceries, because they do not want to do much shopping. This problem is solved through the software solution because the solution will promote recipes which a high Inventory ingredient/ recipe ingredient factor and the structured shopping list might contribute to lower the number of impulse buys.

It can be complicated to plan a meal schedule for many people, but through the software solution the meals can be scheduled for more people, and the shopping list will be updated accordingly. This is also a trait for people living alone, since they on average have a higher food waste level than people living together and therefore the ability to only use a specific amount, will both lower their food waste level as well as the cost of the meals.

Having a diet can be a complication when going shopping, because it can be difficult to find the specific ingredients needed for the specific recipes, by having a shopping list from home, which includes the ingredients of the recipes, the user will much faster be able to find the ingredients.

By not knowing what you already have at home, the food waste level is bound to rise to be higher than if you are keeping track. The software solution allows the user to keep track of everything in an easy manner, by adding items either directly to the inventory list, or by adding them from the shopping list after the user have bought the item. 

the overall purpose of the solution which is planning and managing a meal plan have been showed to lower the food waste level. The functionality of having to shop only for the amount of what you need, also lets the user get a lower food waste level,and keeping track of the inventory makes sure that items already owned are not bought.

The program have the ability to let the user choose a specific diet that they want to follow, these will put items that can not be eaten on the blacklist, making sure that the recipes that the program lets the user plan follows a specific diet.

The interaction design of the program is created using the Scenario-Based design method, which incorporates user stories, conceptual scenarios, design requirements and scenario corpus into a conceptual model and concrete scenarios into paper prototypes. It then assembled the conceptual model and paper prototypes into design language. The design language guideline have been followed throughout the design of the program. The design requirements of the program functionality have been taken into consideration when the software system have been made, though all of the requirements have not been fulfilled.

The structure for the software solution's system is defined by the use of the layer pattern (MVVM) and a distribution pattern variant of the centralized pattern for the process architecture. 