\section{How to avoid food waste} \label{HowToAvoidFoodWaste}
With 20 \% of grocery being bought in a Danish household getting thrown away, finding ways to decrease or prevent food waste is essential. When throwing away food, it is not only groceries that could have been eaten, which are going to waste, but also the energy which have been used to grow, make or harvest the product are being wasted as well.

According to The Danish Ministry of the Environment\cite{madSpild_Notat}, one of the initiatives that can be taken against food waste, is to make recipes which incorporate the leftovers, and other uncooked groceries. By doing this, food that might otherwise go bad, will be used instead of being thrown away.

Another initiative that can be taken according to The Danish Ministry of the Environment, is to sell quantity discount, with the option to receive some of the groceries later. An example would be if one cucumber is sold for 10 DKK, while two is sold for 15 DKK, some people would choose to buy two cucumbers because of the quantity discount, even if they just needed one at the moment. With a "recieve later" option, you would be able to buy two cucumbers, but only receive one at the moment. Then you would be able to come back to the store later to get the other cucumber when you needed it. An initiative like this have already been made by the English supermarket chain Tesco. The initiative is called "Buy One Get One Free Later", and is an initiative for fresh groceries like fruit and vegetables.

To give the consumer better information about when food is going bad is also an initiative that The Danish Ministry of Environment is recommending. By informing consumers of what "Best before", "Last day of sale" and so on means, would get the consumer to use the food more properly. Also by getting the consumer to rely more on smell, taste and looks, rather than just the date of production on the groceries, would lead to lesser food waste.

The Danish Ministry of the Environment also recommends getting discounts even though you don't buy in quantity. Being able to share an offer with another person would let the supermarkets get volumes bought, and the consumer would be able to only acquire the amount necessary.

Only using the groceries necessary is also a way to achieve lesser food waste. Following recipes will help with this if the recipe is specific. If for example the recipe is for 4 persons, and 4 persons will be eating, if the recipe is correct food will not be wasted, and even if there are leftovers after the meal, using these as lunch for the next day, or eating the leftovers the day after, will achieve a lesser waste of food\cite{madSpild_MindreMadspild}.

Keeping track of products that is being thrown out is another way to reduce food waste. If a pattern occurs you may be overbuying food. An example is if you throw out 500 grams of beef each week, you may be overbuying beef, and might as well buy 500 grams of beef less each week\cite{madSpild_Greatist}.

Eating the food which is closest to the expiration date is a good way to prevent food waste as well. If the refrigerator is being filled up as the week goes by, new products might be put in the front as they are bought last. By keeping track of the expiration dates by either organizing the products and having the oldest in the front of the refrigerator, or by writing down the expiration date of the different products, should help on the food waste problem as old food gets eaten first.

According to United States Environmental Protection Agency\cite{madSpild_EPA}, being creative with leftovers is another way of preventing food waste. For example if a loaf of bread has gone stale, cutting it up and using it as croûtons is a great way to reuse it, instead of just throwing it away.

In conclusion; wasting food is a something that we all do, and it is therefore essential to find solutions to minimize food waste. In this section, suggestions have been made, that could help preventing food waste, and therefore help preserving the environment as well. The suggestions have ranged from providing consumers with better information, to providing ideas for supermarkets to help prevent the food waste, that consumers have.