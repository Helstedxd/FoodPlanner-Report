\chapter{Testing}

When the program had been made, some testing where done to find problems within the program. The test included people outside the group, and where conducted with an objective view. The tests ended up with a list of requirements for the program, that need to be changed in order for the users to get the best possible program experience.

\section{UsabilityTest}

To test the usability of the program, a usability test were conducted which included people outside the group, to achieve an objective standpoint of the usability. The usability test was an IDA (Instant Data Analysis) test.

\subsection{Conduction of the test}

The setup for the test consisted of a video camera filming the person, a screen recorder recording what the test participant did within the program, two people describing what the test participant did, and a test leader explaining what the program did, making sure the participant was doing to the task, and helping if the participant got stuck.

When a participant was going to be tested, he was first explained what the program did, and what problems it was supposed to solve. After this the participant was presented the tasks that he was supposed to do within the program, and was told that they should read them out loud, so when reviewing the video, it was possible to know which task was the test participant where executing. The tasks that the participant did can be found in \cref{UsabilityAppendix}.

When the participant had completed all of the given tasks, some follow-up questions were asked. The questions was asked to see if the user had any comments that was not mentioned throughout the execution of the tasks. These questions can as well as the tasks be found in \cref{UsabilityAppendix}.

Five test participants did the tasks, and after all of these were through the test, the analysis could begin. The point of the analysis was to find common errors that the participants had within the program, so these could be changed. The errors was then divided into categories of the following:

\begin{itemize}
    \item Critical problem: Problems that must be changed for the program to work probably.
    \item Serious problems: Problems that can be changed for a better usability.
    \item Cosmetic problems: Small problems that does not need to be changed, though changing them will provide a better usability.
\end{itemize}

\subsection{Test outcome}

When the test had been conducted the problems was identified, the problems was summed up into the following list:

\begin{itemize}
    \item Test participants clicked the day they wanted to plan a meal in the mealplan page, as they found this to be the most intuitive way to plan a meal, instead of going to the search page.
    \item Some test participants found it more intuitive to change the "number of days to shop for" within the shopping list page instead of the settings page. Some of the participants found that a link from the shopping list page to the settings page would be sufficient to fix this problem.
    \item Participants found it hard to know that they needed to click the update button for a planned meal to be updated, when changing the date or the number of meal participants.
    \item Some participants did not know that a meal was deleted from the meal plan, after clicking the subtraction sign, there was not enough feedback.
    \item Some participants mentioned that they had a hard time finding consistency in whether the data was updated automatically or they needed to click an update button.
    \item Some participant mentioned that they had problems seeing how the settings screen were divided. They mentioned that there was too much space between some of the objects within the area of a setting.
    \item When scheduling a recipe, some of the participants mentioned that he top bar was hard to figure out, and the information in this top bar needed to be communicated clearer.
    \item Some of the participants mentioned that there was not enough feedback when adding and ingredient to the rated ingredients list.
    \item Some of the participants thought that having a rating of 100 was too much and a rating of 10 would have been better, where as other thought that 1-100 was a simple and easy scale to use.
\end{itemize}

The list of problems was then analysed and the problems where divided into the categories of critical, serious, and cosmetic.

\subsubsection{Critical Problems}

\begin{itemize}
    \item Not knowing to click the update button when changing information of a scheduled meal.
    \item Not knowing that a meal was deleted after clicking the delete meal button.
    \item Consistency of whether the program updates automatically or if the user have to click to update.
\end{itemize}    
The first critical problem can be solved by letting the program update automatically. The program updates automatically all other places in all other screens, so by letting it update a recipe the moment something is changed and not by clicking an update button, would be come natural for the user.

The second critical problem can be solved by exiting the recipe screen and going to the meal plan, when deleting a meal. THis way the user could see that the recipe was deleted, because it would not be shown in the screen any more, and it would not be shown in the meal plan either.

The last critical problem can be solved in the same way as the first one. The only place in the program that does not automatically update is the changing of a recipe, so by making this update automatically, consistency throughout the program would be achieved.

\subsubsection{Serious Problems}

\begin{itemize}
    \item Trying to plan a meal by clicking the day in the mealplan page, instead of going to the search page.
    \item Finding it hard to know how the top bar in the recipe screen worked.
    \item Changing the number of days to shop for within the shopping list page instead of the settings page.
\end{itemize}

The first serious problem can be solved by implementing a method that lets the user go to the search page when clicking a specific date, and when the user then clicks finds the recipe they want scheduled, the date that it is set to by default should then be the date they clicked in the mealplan screen.

The second serious problem can be solved by making clearer icons. The participants of the test had no problems with deleting a recipe as the subtraction sign is a common sign for deleting something, the same goes for adding a recipe as the addition sign is a common sign for adding something. In the critical problems it is described that the program should update automatically in the recipe screen, so the update button could be deleted. If a label with the text "Participants" were placed left for the participants box, then the user would know that these where participants, and in this way the second serious problem could be solved.

The third serious problem is easily solvable, as this can be done by moving the ability to change the number of days to shop for to the shopping list page instead of the settings page.

\subsubsection{Cosmetic problems}

\begin{itemize}
    \item Finding it hard to see how the setting screen were divided.
    \item Not getting enough feedback when adding an ingredient to the rated ingredient list.
    \item Having a 1-100 rating is too much, 1-10 is enough.
\end{itemize}

The first cosmetic problem can be solved by deleting some of the blank space in the settings screen. An example could be to move the buttons up by the side of the listview, as the buttons take up all the horizontal space it is in right now.

The second cosmetic problem can be solved by letting a label say when the item has been have been added to the rated ingredients list.

The third cosmetic problem can be solved by replacing the rating system by 10 star rating system.

\subsection{Conclusion}

The program has usability problems that can be solved. These problems are divided into all the three categories, critical, serious, and cosmetic. The problems are equally divided into the three categories, so some of the problems are important to fix if the program was to be released to the public.