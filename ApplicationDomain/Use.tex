\chapter{Usage}
This chapter will describe the systems interaction with its surroundings.

\section{Overview}
In this section an actor specification will be presented followed by a user pattern diagram showcased in state machine.

\subsection{Actor specification}
An actor will be described by a actor specification \ref{Actor specification}, the actor is a user of the system.
\vline
\textbf{\textit{User}}\label{Actor specification}
\textbf{Objective:} A person who prepare their meals them self, either for them self or for a family household. The user's primary need is to plan their their week in order to efficiently shop their groceries and lessen food waste.
\textbf{characteristic:} The system includes a user base of which the users have different needs and preferences.
\textbf{Examples:} User A is an allergic and has to make meals which excludes nuts. When A is shopping she needs to look at the package info to make sure that the product does not contain trace of nuts.

User B is a young student who is in a relationship. User B has a tight budget and a little time frame to shop in. B finds it difficult to sustain an overview of their share storage of food. Therefore B will sometimes buy products they do not need, for example, B might purchase milk even though they already have three litres stored. This will sometime result in them not being able to use all the milk before it expires, which is bad for their shared budget and food waste.
\vline

\subsection{User pattern}
The user pattern shows an abstraction between the interactions i the system and actor. To show these interactions an state machine diagram have been chosen. The diagram shows the how the dynamic states can shift trough interactions with the actor. Even tough many of the details is excluded, it still gives a good overview of how the logic i set up in the user pattern and how the the dynamic flow goes. 

The purpose of the diagram is to create an overview of the application domains interactions with the system. This will be used to find requirrements for the functions and user interface.

[Indsæt figur som viser brugsmønsteret, men vendt til efter forelæsningen så Anders kan sikre at det ser rigitgt ud.]