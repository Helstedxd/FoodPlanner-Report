\section{Food Waste}
Every year danish households throws away nearly 237,000 tons of food\cite{maSpild_iTal}, that could have been eaten by people. This is estimated to be nearly 16 mio. DKK. About 20 \% of a danish households food budget is wasted on throwing away food that could have been eaten, this means that if no food were wasted at all, 1 mio. people could be fed, nearly corresponding to 20 \% of the danish population.

The people who throws away most food, are the people living alone. The average person living alone throws away 98.8 kg of food per year, whereas a household with 5 people throws away 46.8 kg of food per person per year. It is mostly dairy products, vegetables and bread that gets thrown away, though in December a lot of meat is also thrown away, due to the fact that it is Christmas and at this time people eat more food.

Since 2007 the danish people have been purchasing 10 \% less food, a lot of this food used to end up as food waste. 54 \% of the danish people rarely or never uses the food from the night before as leftovers, and 69 \% does not use leftovers for lunch the day after.

20 \% of danish people does not make a shopping list before shopping for groceries. 52 \% of people does not make a meal plan, 31 \% does it sometimes and 17 \% does it often.

81 \% of danish people would use leftovers to avoid food waste, 50 \% would make a meal plan and make a grocery shopping list. 59 \% of the consumers believe that the danish food waste is their own fault.\cite{maSpild_iTal}
