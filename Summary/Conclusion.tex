\chapter{Conclusion}

It is a problem that a consumer wants fresh groceries, and at the same time purchases too many groceries, because they do not want to do much shopping. This problem is solved through the software solution because the user will not buy too much food, because the solution only tells the user to buy as much as is needed for the recipes scheduled.

It can be hard to plan a meal schedule for many people, but through the software solution the meals can be scheduled for more people, and the shopping list will be update accordingly. This is also a trait for people living alone, since they on average have a higher food waste level than people living together and therefore the ability to only use a specific amount, will both lower their food waste level as well as the cost of the meals.

Having a diet can be a problem when going shopping, because it can be hard to find the specific ingredients needed for the specific recipes, by having a shopping list from home, which includes the ingredients of the recipes, the user will much faster be able to find the ingredients.

By not knowing what you already have at home, the food waste level is bound to rise to be higher than if you are keeping track. The software solution allows the user to keep track of everything in an easy manner, by adding items either directly to the inventory list, or by adding them from the shopping list after the user have bought the item. 

The way that the food waste level have been lowered through the software solution, is by letting the user make a meal schedule, as this have been showed to lower the food waste level. The functionality of having to shop only for the amount of what you need, also lets the user get a lower food waste level,and keeping track of the inventory makes sure that items already owned are not bought.

The program have the ability to let the user choose a specific recipe that they want to follow, these will put items that can not be eaten on the blacklist, making sure that the recipes that the program lets the user schedule follows a specific diet.

The usability design of the program is created using a Scenario-Based design model, which incorporates user stories, conceptional scenarios, concrete scenario, and paper prototyping. The aforementioned sections ended with a list of design requirements and a conceptual model, which where used to create design language. The design language have been fulfilled throughout the design of the program. The design requirements of the program functionality have been taken into consideration when the software system have been made, though all of the requirements have not been fulfilled.