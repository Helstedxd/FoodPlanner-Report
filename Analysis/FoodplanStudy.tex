\section{Meal Plan Study}
\textit{Stop Spild Af Mad}\cite{madSpild_RapportAdfaerd} has conducted a study where eight participants had to create and follow a weekly meal plan.
Each participant had to create a meal plan, in which they could freely choose the different recipes with inspiration from cookbooks and/or the Internet.

\textbf{Planning:}
The participants had some trouble with pulling themselves together and getting started creating the meal plan, but they were happy to have it, once it was done. It could be beneficial for the person that were in charge of grocery shopping to be the one making the meal plan, since the involvement of other family members could increase the planning time.

\textbf{Modules:}
It was almost impossible for the participants to follow the meal plan for a whole week without having to reorganize or modify it. Most of the participants made small modules that could be moved around, depending on the family's situation. It was optimal with modules spanning two to three days. Varied meals were preferred above reheated meals.

\textbf{Grocery shopping:}
Seven out of the eight participants were shopping groceries almost every day. This was not seen as a nuisance, but after having planned grocery shopping for a whole week, they experienced a big relief in their everyday life. This was expressed in terms of more personal freedom as a perk of saving time and money. Planning ahead also decreased buying on impulse.

\subsection{Reflection} \label{FoodPlanStudyReflection}
The participant had trouble getting started creating a new food plan for the week. They also wanted a more realistic meal plan instead of having to create it from cookbooks. Realistic in the sense, that the plan should fit into their context. A solution to this problem might be solved by automating a lot of the tasks involved with having to come up with the different recipes and do the grocery shopping.
By having a program come up with suggested meals based on different aspects such as the:
\begin{itemize}
\item Food you already have
\item Food you like
\item Price
\item Stock of groceries in stores nearby
\end{itemize}
People could quickly plan a schedule or even have the program suggest one for them.